\documentclass[10pt,landscape]{article}
\usepackage{multicol} % multiple columns
\usepackage[landscape]{geometry} % landscape
\geometry{top=1cm,left=1cm,right=1cm,bottom=1cm} % margins
\pagenumbering{gobble} % remove page numeration
\usepackage{lipsum}% dummy text
\usepackage{parskip} % no indentation

\begin{document}
\begin{center}
\textbf{\Large Econometrics CheatSheet}
\end{center}

\begin{multicols}{3}
\section*{Econometrics Theory}
\subsection*{Definition}
\textbf{Econometrics} - social science with the objective of quantify the relationships between economic agents.
\textbf{Econometric model} - simplification of the reality to explain economic phenomena.

\subsection*{Phases of an econometric model}
\begin{enumerate}
	\item Specification
	\item Estimation
	\item Validation
	\item Utilization
\end{enumerate}

\section*{Regression Analysis}
Study and predict the mean value of a variable regarding the base of fixed values of other variables.

We usually use Ordinary Least Squares (OLS).


\section*{Correlation Analysis}
The correlation analysis not distinguish between dependent and independent variables.

\textbf{Simple Correlation}

Measure the grade of lineal association between two variables.



\section*{Utilization}
\subsection*{Interpretation of the model}


\begin{tabular}{ | c | c | c | c | }
	\hline
	Model & Dependent & Independent & Interpretation $\beta_1$ \\
	\hline
	Level-level & $y$ & $x$ & $\Delta y = \beta_1 \Delta x$ \\ 
	\hline
	Level-log & $y$ & cell6 & cell2 \\  
	\hline
	Log-level & cell8 & cell9 & cell1 \\
	\hline
	Log-log & cell10 & cell1 & cell1 \\
	\hline
\end{tabular}


\lipsum
\end{multicols}
\end{document}
