\documentclass[10pt,landscape]{article}

% packages:
\usepackage{multicol} % for multiple columns
\usepackage[landscape]{geometry} % for landscape
\usepackage{parskip} % remove text indentation
\usepackage{graphicx} % for scale tables
\usepackage{lipsum}% for dummy text (will be removed int he final version)

% page customization
\geometry{top=1cm,left=1cm,right=1cm,bottom=1cm} % margins configuration
\pagenumbering{gobble} % remove page numeration

% -----document-----
\begin{document}
% title
\begin{center}
\textbf{\Large Econometrics CheatSheet}
\end{center}
% content
\begin{multicols}{3} % set page columns to 3

\section*{Basic concepts}

\subsection*{Definition of econometrics}

\textbf{Econometrics} - is a social science discipline with the objective of quantify the relationships between economic agents, contrast economic theories and evalue and implement government and business policies.

\textbf{Econometric model} - is a simplificated representation of the reality to explain economic phenomena.

\subsection*{Data types}

\begin{enumerate}
\item Cross section: data taken at a given moment in time, an static "photo". Order does not matter.
\item Temporal series: observation of one/many vairable/s across time. Order does matter.
\item Panel data: consist of a temporal serie for each observation of a cross section.
\item Pooled cross sections: combines cross sections from different temporal periods.
\end{enumerate}

\subsection*{Phases of an econometric model}

\begin{enumerate}
\item Specification
\item Estimation
\item Validation
\item Utilization
\end{enumerate}

\subsection*{Assumptions of the econometric model}
Under this assumptions the estimators of the parameters will present "good properties". GAUSS MARKOV ASSUMPTIONS (EXTENDED)
\begin{itemize}
\item Parameters linearity.
\item The sample of the poblation is random. Caracteristics:
\begin{itemize}
\item Independence: independence, that guarantees that all the covariances between independents are zero.
\item Identical distribution: that guarantees that the $n$ expected values and variances of the observations are the same.
\end{itemize}
\item $E(\epsilon / X_1, X_2, ..., X_k) = 0$, guarantees that the estimations are unbiased, that have some implications:
\begin{itemize}
\item $E(\epsilon) = 0$ there are none systematic errors.
\item $Cov(\epsilon, X_1) = Cov(\epsilon, X_2) = ... = Cov(\epsilon, X_k) = 0$ there are no relevant variables not included in the model.
\item $E(Y/X_1, X_2, ..., X_k) = \beta_0 + \beta_1 X_1 + \beta_k X_k$ the lineal relation between $Y$ and $X_1, ..., X_k$ is fulfilled, at least in average.
\end{itemize}
\item Homocedasticity: $Var(\epsilon_i / X_{1i}, X_{2i}, ..., X_{ki}) = \sigma^2$, the variability of the error is the same for all levels of $x$. Guarantees that the estimations are efficient. Implies that: $Var(Y_i / X_{1i}, X_{2i}, ..., X_{ki}) = \sigma^2$, the variability of the dependent variable is the same for all levels of $x$.
\item No autocorrelation: $Cov(\epsilon_i \epsilon_j = 0 \rightarrow Cov(Y_i Y_j / X) = 0$ for every $i$ different from $j$. The errors do not contain information about other errors.
\item The distribution of the errors is normal (is not always necessary).
\item No multicolineality: none of the independent variables is constant nor exist an exact (or aproximate) linear relation between them, they are linearly independents.
\item The number of available data is greater than $k+1$ ($\beta$ parameters to estimate).
\end{itemize}

The homocedasticity and no autocorrelation asumptions can also be written in matrix form: $Var(\epsilon / X) = \sigma^2 I_n$

\subsection*{Interpretation of the coefficients}
\scalebox{0.8}{
\begin{tabular}{ | c | c | c | c | }
	\hline
	Model & Dependent & Independent & Interpretation $\beta_1$ \\
	\hline
	Level-level & $y$ & $x$ & $\Delta y = \beta_1 \Delta x$ \\ 
	\hline
	Level-log & $y$ & $log(x)$ & $\Delta y = (\beta_1/100)[1 \% \Delta x]$ \\  
	\hline
	Log-level & $log(y)$ & $x$ & $\% \Delta y = (100 \beta_1) \Delta x$ \\
	\hline
	Log-log & $log(y)$ & $log(x)$ & $\% \Delta y = \beta_1 \% \Delta x$ \\
	\hline
	Quadratic & $y$ & $x + x^2$ & $\Delta y = (\beta_1 + 2 \beta_2 x) \Delta x$ \\
	\hline
\end{tabular}
}




\section*{Regression Analysis}
Study and predict the mean value of a variable regarding the base of fixed values of other variables.
We usually use Ordinary Least Squares (OLS).

\section*{Correlation Analysis}
The correlation analysis not distinguish between dependent and independent variables.
\textbf{Simple Correlation}
Measure the grade of lineal association between two variables.

\section*{Utilization}
\subsection*{Interpretation of the model}


\section*{Heterocedasticity}
The residuals $u_i$ of the poblational regression function don't have the same variance $\sigma^2$:

$Var(u_i \mid x_i) = \sigma_i^2; i = 1, ..., n$

\subsection*{Consequences}
Under the Gauss-Markov Theorem asumptions, OLS estimators are not efficient. The estimations of the variance of the estimators are biased. The hyphotesis contrast and the confidence intervals are not reliable.
\subsection*{Detection}
Plots (look for structures in plots with the square residuals) and contrasts: Park test, Goldfield-Quandt, Bartlett, Breush-Pagan, CUSUMQ, Spearman, White.
White's null hypothesis:

$H_0 = HOMOCEDASTICITY$
\subsection*{Correction}
\begin{itemize}
	\item When the variance structure is known, use weighted least squares.
	\item When the variance structure is not known: make asumptions of the possible structure and apply weighted least squares
	\item Supossing that $\sigma_i^2$ is proportional to $x_i^2$, divide by $x_i$
\end{itemize}
\lipsum

\end{multicols}
\end{document}
