\documentclass[10pt, a4paper, landscape]{extarticle}

% -----packages-----
\usepackage{multicol} % for multiple columns
\usepackage[landscape]{geometry} % for landscape
\usepackage{parskip} % remove text indentation
\usepackage{graphicx} % for scale tables
\usepackage{enumitem} % indent of lists
\usepackage{tikz} % for plots
\usetikzlibrary{patterns} % for patterns
\usepackage{hyperref} % for hyperlinks
\usepackage{amsmath} % for writing normal text on equations
\usepackage{scrlayer-scrpage} % page foot
\usepackage[compact]{titlesec} % titles spacing

% -----page customization-----
\geometry{top=1cm,left=1cm,right=1cm,bottom=1cm} % margins configuration
\pagenumbering{gobble} % remove page numeration
\setlength{\parskip}{0cm} % paragraph skip length
% title spacing:
\titlespacing{\section}{0pt}{2ex}{1ex}
\titlespacing{\subsection}{0pt}{1ex}{0ex}
\titlespacing{\subsubsection}{0pt}{0.5ex}{0ex}

% -----document-----
\begin{document}

\cfoot{\href{https://github.com/marcelomijas/econometrics-cheatsheet}{\normalfont \footnotesize Version ts1.0-en - github.com/marcelomijas/econometrics-cheatsheet}}
\setlength{\footskip}{12pt}

\begin{multicols}{3} % set columns to 3

\begin{center}
	\textbf{\LARGE \href{https://github.com/marcelomijas/econometrics-cheatsheet}{Time Series Cheat Sheet}}
	\\ {\footnotesize By Marcelo Moreno - King Juan Carlos University} 
	\\ {\footnotesize As part of the Econometrics Cheat Sheet Project}
\end{center}

\section*{Basic concepts}
	\subsection*{Definitions}
		\textbf{Time series} - is a succession of quantitative observations of a phenomena ordered in time.
		\\ There are some variations of time series:
		\begin{itemize}[leftmargin=*]
			\item \textbf{Panel data} - consist of a time series for each observation of a cross section.
			\item \textbf{Pooled cross sections} - combines cross sections from different time periods.
		\end{itemize}
		\textbf{Stochastic process} - sequence of random variables that are indexed in time.
	\subsection*{Components of a time series}
		\begin{itemize}[leftmargin=*]
			\item \textbf{Trend} - is the long-term general movement of a series.
			\item \textbf{Seasonal variations} - are periodic oscillations that are produced in a period equal or inferior to a year, and can be easily identified on different years (usually are the result of climatology reasons).
			\item \textbf{Cyclical variations} - are periodic oscillations that are produced in a period greater than a year (are the result of the economic cycle).
			\item \textbf{Residual variations} - are movements that do not follow a recognizable periodic oscillation (are the result of eventual non-permanent phenomena that can affect the studied variable in a given moment).
		\end{itemize}
	\subsection*{Type of time series models}
		\begin{itemize}[leftmargin=*]
			\item \textbf{Static models} - the relation between $y$ and $x$'s is contemporary. Conceptually:
			\begin{center}
				$y_t = \beta_0 + \beta_1 x_t + u_t$
			\end{center}
			\item \textbf{Distributed-lag models} - the relation between $y$ and $x$'s is not contemporary. Conceptually:
			\begin{center}
				$y_t = \beta_0 + \beta_1 x_t + \beta_2 x_{t-1} + ... + \beta_{s+1} x_{t-s} + u_t$
			\end{center}
			The long term cumulative effect in $y$ when $\Delta x$ is:
			\begin{center}
			 	$\beta_1 + \beta_2 + ... + \beta_{s+1}$
			\end{center}
		 	\item \textbf{Dynamic models} - a temporal drift of the dependent variable is part of the independent variables (endogeneity). Conceptually:
		 	\begin{center}
		 		$y_t = \beta_0 + \beta_1 y_{t-1} + ... + \beta_s y_{t-s} + u_t$
		 	\end{center}
	 		\item Combinations of the above, like the rational distributed-lag models (distributed-lag + dynamic).
		\end{itemize}

\columnbreak

\section*{Assumptions and properties}
	\subsection*{OLS model assumptions under time series}
		Under this assumptions, the estimators of the OLS parameters will present good properties. \textbf{Gauss-Markov assumptions extended applied to time series}:
		\begin{enumerate}[leftmargin=*, label=ts\arabic*.]
			\item \textbf{Parameters linearity and weak dependence}.
			\begin{enumerate}[leftmargin=*, label=\alph*.]
				\item $y_t$ must be a linear function of the $\beta$'s.
				\item The stochastic $\lbrace(x_t, y_t): t = 1, 2, ..., T\rbrace$ is stationary and weakly dependent.
			\end{enumerate} 
			\item \textbf{No perfect collinearity}.
			\begin{itemize}[leftmargin=*]
				\item There are no independent variables that are constant: $Var(x_j) \neq 0$
				\item There is not an exact linear relation between independent variables.
			\end{itemize}
			\item \textbf{Conditional mean zero and correlation zero}.
			\begin{enumerate}[leftmargin=*, label=\alph*.]
				\item There are no systematic errors: $E(u_t | x_{1t}, ..., x_{kt}) = E(u_t) = 0 \rightarrow$ \textbf{strong exogeneity} (a implies b).
				\item There are no relevant variables left out of the model: $Cov(x_{jt} , u_t) = 0$ for any $j = 1, ..., k \rightarrow$ \textbf{weak exogeneity}.
			\end{enumerate}
			\item \textbf{Homoscedasticity}. The variability of the residuals is the same for any $x$: $Var(u_t | x_{1t}, ..., x_{kt}) = \sigma^2$
			\item \textbf{No auto-correlation}. The residuals do not contain information about other residuals: $Corr(u_t, u_s | x) = 0$ for any given $t \neq s$.
			\item \textbf{Normality}. The residuals are independent and identically distributed (\textbf{i.i.d.} so on): $u \sim N(0,\sigma^2)$
			\item \textbf{Data size}. The number of observations available must be greater than $(k + 1)$ parameters to estimate. (IS already satisfied under asymptotic situations)
		\end{enumerate}	
	\subsection*{Asymptotic properties of OLS}
		Under the econometric model assumptions and the Central Limit Theorem:
		\begin{itemize}[leftmargin=*]
			\item Hold (1) to (3a): OLS is \textbf{unbiased}. $E(\hat{\beta}_j) = \beta_j$
			\item Hold (1) to (3): OLS is \textbf{consistent}. $plim(\hat{\beta}_j) = \beta_j$ (to (3b) left out (3a), weak exogeneity, biased but consistent)
			\item Hold (1) to (5): \textbf{asymptotic normality} of OLS (then, (6) is necessarily satisfied): $u \sim_a N(0,\sigma^2)$.
			\item Hold (1) to (5): \textbf{unbiased estimate of $\sigma^2$}. $E(\hat{\sigma}^2) = \sigma^2$
			\item Hold (1) to (5): OLS is \textcolor{blue}{\href{https://www.youtube.com/watch?v=68ugkg9RePc}{BLUE}} (Best Linear Unbiased Estimator) or \textbf{efficient}. 
			\item Hold (1) to (6): hypothesis testing and confidence intervals can be done reliably.
		\end{itemize}

\columnbreak

\section*{Trends and seasonality}
	\textbf{Spurious regression} - is when the relation between $y$ and $x$ is due to factors that affect $y$ and have correlation with $x$, $Corr(x, u) \neq 0$. Is the \textbf{non-fulfillment of ts3}.
	\subsection*{Trends}
		Two time series can have the same (or contrary) trend, that should lend to a high level of correlation. This, can provoke a false appearance of causality, the problem is in what is known as \textbf{spurious regression}. Given the model:
		\begin{center}
			$y_t = \beta_0 + \beta_1 x_t + u_t$
		\end{center}
		where:
		\begin{center}
			$y_t = \alpha_0 + \alpha_1 Trend + v_t$
			\\ $x_t = \gamma_0 + \gamma_1 Trend + v_t$
		\end{center}
		Adding a trend to the model can solve the problem:
		\begin{center}
			$y_t = \beta_0 + \beta_1 x_t + \beta_2 Trend + u_t$
		\end{center}
		The trend can be linear or non-linear (quadratic, cubic, exponential, etc.)
		\\ Another way is making use of the \textbf{Hodrick-Prescott filter} to extract the trend (smooth) and the cyclical component.
	\subsection*{Seasonality}
		\setlength{\multicolsep}{0pt} % reduce vertical spacing betwen subsection and multicols
		\begin{multicols}{2} % set columns to 2
		A time series with can manifest seasonality. That is, the series is subject to a seasonal variations or pattern, usually related to climatology conditions.
		\\ For example, GDP (black) is usually higher in summer and lower in winter. Seasonally adjusted series ({\color{red} red}) for comparison.
		\columnbreak
			\begin{tikzpicture}[scale=0.18]
				\draw [step=2, gray, very thin] (-10,-10) grid (10,10); % background grid
				\draw [thick, <->] (-10,10) node [anchor=south] {$y$} -- (-10,-10) -- (10,-10) node [anchor=north] {$t$}; %axis
				\draw [black]
				(-8.0000,-7.2061) -- 
				(-7.5676,-5.1908) -- 
				(-7.1351,-8.0000) -- 
				(-6.7027,-2.3817) -- 
				(-6.2703,-3.9695) -- 
				(-5.8378,-1.1603) -- 
				(-5.4054,-4.5802) -- 
				(-4.9730,0.8550) -- 
				(-4.5405,-1.5267) -- 
				(-4.1081,-0.3053) -- 
				(-3.6757,-4.5191) -- 
				(-3.2432,-0.4885) -- 
				(-2.8108,-2.3206) -- 
				(-2.3784,-0.4275) -- 
				(-1.9459,-4.2137) -- 
				(-1.5135,0.3664) -- 
				(-1.0811,-1.7099) -- 
				(-0.6486,-0.5496) -- 
				(-0.2162,-4.3969) -- 
				( 0.2162, 1.0992) -- 
				(0.6486,-1.0382) -- 
				(1.0811,1.2824) -- 
				(1.5135,-2.8702) -- 
				(1.9459,1.7099) -- 
				(2.3784,-1.0382) -- 
				(2.8108,1.5267) -- 
				(3.2432,-1.8321) -- 
				(3.6757,3.3588) -- 
				(4.1081,1.0992) -- 
				(4.5405,4.2137) -- 
				(4.9730,0.9160) -- 
				(5.4054,6.2901) -- 
				(5.8378,4.3969) -- 
				(6.2703,6.5954) -- 
				(6.7027,3.4198) -- 
				(7.1351,8.0000) -- 
				(7.5676,6.1069) -- 
				(8.0000,6.9008);
				\draw [red] 
				(-8.0000,-6.2061) -- 
				(-7.5676,-6.0018) -- 
				(-7.1351,-6.1000) -- 
				(-6.7027,-5.0817) -- 
				(-6.2703,-3.9095) -- 
				(-5.8378,-3.0603) -- 
				(-5.4054,-3.0002) -- 
				(-4.9730,-2.4550) -- 
				(-4.5405,-2.5267) -- 
				(-4.1081,-2.3053) -- 
				(-3.6757,-2.5191) -- 
				(-3.2432,-2.4885) -- 
				(-2.8108,-2.3206) -- 
				(-2.3784,-2.4275) -- 
				(-1.9459,-2.2137) -- 
				(-1.5135,-1.6664) -- 
				(-1.0811,-2.0099) -- 
				(-0.6486,-1.8496) -- 
				(-0.2162,-1.3969) -- 
				(0.2162,-1.0992) -- 
				(0.6486,-1.0382) -- 
				(1.0811,-1.2824) -- 
				(1.5135,-1.0002) -- 
				(1.9459,-0.8099) -- 
				(2.3784,-0.6382) -- 
				(2.8108,-0.6267) -- 
				(3.2432,0.6321) -- 
				(3.6757,1.0588) -- 
				(4.1081,1.3992) -- 
				(4.5405,2.2137) -- 
				(4.9730,2.5160) -- 
				(5.4054,3.5901) -- 
				(5.8378,4.0969) -- 
				(6.2703,5.2954) -- 
				(6.7027,5.3198) -- 
				(7.1351,6.2000) -- 
				(7.5676,6.9069) -- 
				(8.0000,6.6008);
			\end{tikzpicture}
		\end{multicols}
		\begin{itemize}[leftmargin=*]
			\item This problem is part of what is known as \textbf{spurious regression}. A seasonal adjustment can solve the problem.
		\end{itemize}
		A simple \textbf{seasonal adjustment} could be creating stationary binary variables and adding them to the model. For example, for quarterly series ($QX_t$ are binary variables):
		\begin{center}
			$y_t = \beta_0 + \beta_1 Q2_t + \beta_2 Q3_t + \beta_3 Q4_t + \beta_4 x_{1t} + ... + \beta_k x_{kt} + u_t$
		\end{center}
		Another way is to seasonally adjust (sa) the variables, and then, do the regression with the adjusted variables:
		\begin{center}
			$z_t = \beta_0 + \beta_1 Q2_t + \beta_2 Q3_t + \beta_3 Q4_t  + v_t \rightarrow \hat{v}_t + E(z_t) = \hat{z}_t^{sa}$
			$\hat{y}_t^{sa} = \beta_0 + \beta_1 \hat{x}_{1t}^{sa} + ... + \beta_k \hat{x}_{kt}^{sa} + u_t$
		\end{center}
		There are much better and complex methods to seasonally adjust a time series, like the \textbf{X-13ARIMA-SEATS}.
		
\columnbreak

\section*{Auto-correlation}
	The residual of any observation, $u_t$, is correlated with the residual of any other observation. The observations are not independent.
	\begin{center}
		$Corr(u_t, u_s | x) \neq 0$ for any $t \neq s$
	\end{center}
	The ``natural" context of this phenomena is time series. Is the \textbf{non-fulfillment of ts5}.
	\subsection*{Consequences}
		\begin{itemize}[leftmargin=*]
			\item OLS estimators still are unbiased.
			\item OLS estimators still are consistent.
			\item OLS is \textbf{not efficient} anymore, but still a LUE (Linear Unbiased Estimator).
			\item \textbf{Variance estimations of the estimators are biased}: the construction of confidence intervals and the hypothesis contrast are not reliable.
		\end{itemize}
	\subsection*{Detection}
		\begin{itemize}[leftmargin=*]
			\item \textbf{Scatter plots} - look for scatter patterns on $u_{t-1}$ vs. $u_t$.
			\setlength{\multicolsep}{0pt} % reduce vertical spacing betwen subsection and multicols
			\setlength{\columnsep}{6pt} % increment spacing between columns
			\begin{multicols}{3} % set columns to 3
				\begin{center}
					\textbf{\footnotesize Ac.}
				\end{center}
				\vspace{2.0pt}
				\begin{tikzpicture}[scale=0.11]
					\draw [step=2, gray, very thin] (-10,-10) grid (10,10); % grid
					\draw [thick,->] (-10,0) -- (10,0) node [anchor=south] {$u_{t-1}$}; % ut-1 axis
					\draw [thick,-] (-10,-10) -- (-10,10) node [anchor=west] {$u_t$}; % ut axis
					\draw plot [only marks, mark=*, mark size=6, domain=-8:8, samples=50] (\x,{rnd * 6 + (-2 * (\x)^2 + 40) * 0.1}); % data points
					\draw [thick, dashed, red, -latex] plot [domain=-8:8] (\x,{3 + (-2 * (\x)^2 + 40) * 0.1}); % red arrow
				\end{tikzpicture}
				
			\columnbreak
				
				\begin{center}
					\textbf{\footnotesize Ac.(+)}
				\end{center}
				\begin{tikzpicture}[scale=0.11]
					\draw [step=2, gray, very thin] (-10,-10) grid (10,10); % grid
					\draw [thick,->] (-10,0) -- (10,0) node [anchor=north] {$ u_{t-1}$}; % ut-1 axis
					\draw [thick,-] (-10,-10) -- (-10,10) node [anchor=west] {$u_t$}; % ut axis
					\draw plot [only marks, mark=*, mark size=6, domain=-8:8, samples=25] (\x,{rnd * 6 + 0.5 * \x - 3}); % data points
					\draw [thick, dashed, red, -latex] plot [domain=-8:8] (\x,{3 + 0.5 * \x - 3}); % red arrow
				\end{tikzpicture}
				
				\columnbreak
				
				\begin{center}
					\textbf{\footnotesize Ac.(-)}
				\end{center}
				\begin{tikzpicture}[scale=0.11]
					\draw [step=2, gray, very thin] (-10,-10) grid (10,10); % grid
					\draw [thick,->] (-10,0) -- (10,0) node [anchor=south] {$u_{t-1}$}; % ut-1 axis
					\draw [thick,-] (-10,-10) -- (-10,10) node [anchor=west] {$u_t$}; % ut axis
					\draw plot [only marks, mark=*, mark size=6, domain=-8:8, samples=25] (\x,{rnd * 6 - 0.5 * \x - 3}); % data points
					\draw [thick, dashed, red, -latex] plot [domain=-8:8] (\x,{3 - 0.5 * \x - 3}); % red arrow
				\end{tikzpicture}
			\end{multicols}
			
			\begin{multicols}{2} % set columns to 2
				\item\textbf{Correlogram} - composed of the auto-correlation function (ACF) and the partial ACF (PACF).
				
				\columnbreak
				
				\begin{itemize}[leftmargin=*]
					\item Y axis: correlation [-1,1].
					\item X axis: lag number.
					\item Blue lines: $\pm 1.96/T^{0.5}$
				\end{itemize}
			\end{multicols}
			\begin{center}
				\begin{tikzpicture}[scale=0.22]
					\node at (17.5,23) {\tiny \textbf{ACF}};
					\node at (-1.5,22) {\tiny 1};
					\node at (-1.5,17) {\tiny 0};
					\node at (-1.5,12) {\tiny -1};
					\draw [step=2, gray, very thin] (0,12) rectangle (35,22); % acf
					\draw [thick,|->] (0,22) -- (0,12) -- (35,12); % axis
					\fill [red] (1.5,17)  rectangle  (2,21.9);
					\fill [red] (3.5,17)  rectangle  (4,21.8);
					\fill [red] (5.5,17)  rectangle  (6,21.7);
					\fill [red] (7.5,17)  rectangle  (8,21.6);
					\fill [red] (9.5,17)  rectangle (10,21.5);
					\fill [red] (11.5,17) rectangle (12,21.5);
					\fill [red] (13.5,17) rectangle (14,21.4);
					\fill [red] (15.5,17) rectangle (16,21.3);
					\fill [red] (17.5,17) rectangle (18,21.2);
					\fill [red] (19.5,17) rectangle (20,21);
					\fill [red] (21.5,17) rectangle (22,20.9);
					\fill [red] (23.5,17) rectangle (24,20.7);
					\fill [red] (25.5,17) rectangle (26,20.5);
					\fill [red] (27.5,17) rectangle (28,20.3);
					\fill [red] (29.5,17) rectangle (30,20.1);
					\fill [red] (31.5,17) rectangle (32,19.8);
					\fill [red] (33.5,17) rectangle (34,19.6);
					\draw [blue, very thin] (0,18.5) -- (35,18.5); % above blue line
					\draw [dashed, very thin] (0,17) -- (35,17); % center blue dashed line
					\draw [blue, very thin] (0,15.5) -- (35,15.5); % below blue line
					
					\node at (17.5,11) {\tiny \textbf{PACF}};
					\node at (-1.5,10) {\tiny 1};
					\node at (-1.5,5) {\tiny 0};
					\node at (-1.5,0) {\tiny -1};
					\draw [step=2, gray, very thin] (0,0) rectangle (35,10); % pacf
					\draw [thick,|->] (0,10) -- (0,0) -- (35,0); % axis
					\fill [red] (1.5,5)  rectangle  (2,9.9);
					\fill [red] (3.5,5)  rectangle  (4,4.8);
					\fill [red] (5.5,5)  rectangle  (6,4.6);
					\fill [red] (7.5,5)  rectangle  (8,4.5);
					\fill [red] (9.5,5)  rectangle (10,4.5);
					\fill [red] (11.5,5) rectangle (12,4.7);
					\fill [red] (13.5,5) rectangle (14,4.5);
					\fill [red] (15.5,5) rectangle (16,4.8);
					\fill [red] (17.5,5) rectangle (18,4.6);
					\fill [red] (19.5,5) rectangle (20,4.9);
					\fill [red] (21.5,5) rectangle (22,4.7);
					\fill [red] (23.5,5) rectangle (24,4.6);
					\fill [red] (25.5,5) rectangle (26,4.8);
					\fill [red] (27.5,5) rectangle (28,4.9);
					\fill [red] (29.5,5) rectangle (30,4.8);
					\fill [red] (31.5,5) rectangle (32,4.8);
					\fill [red] (33.5,5) rectangle (34,4.9);
					\draw [blue, very thin] (0,6.5) -- (35,6.5); % above blue line
					\draw [dashed, very thin] (0,5) -- (35,5); % center blue dashed line
					\draw [blue, very thin] (0,3.5) -- (35,3.5); % below blue line
				\end{tikzpicture}
			\end{center}
			Conclusions differ between auto-correlation process.
			\begin{itemize}[leftmargin=*]
				\item \textbf{MA($q$) process}. \textbf{ACF}: only the first $q$ coefficients are significant. The rest are abruptly canceled. \textbf{PACF}: attenuated exponential fast decay or sine waves.
				\item \textbf{AR($p$) process}. \textbf{ACF}: attenuated exponential fast decay or sine waves. \textbf{PACF}: only the first $p$ coefficients are significant. The rest are abruptly canceled.
				\item \textbf{ARMA($p$,$q$) process}. \textbf{ACF}: the coefficients are not abruptly canceled and presents a fast decay. \textbf{PACF}: the coefficients are not abruptly canceled and presents a fast decay.
			\end{itemize}
			If the ACF coefficients do not decay rapidly, there is a clear indicator of lack of
			stationarity in mean, which would lead to take first differences in the
			original series.
			\item \textbf{Formal tests} - $H_0$: No auto-correlation. 
			\\ Supposing that $u_t$ follows an AR(1) process:
			\begin{center}
				$u_t = \rho_1 u_{t-1} + \varepsilon_t$
			\end{center}
			where $\varepsilon_t$ is white noise.
			\begin{itemize}[leftmargin=*]
				\item \textbf{AR(1) t test} (exogenous regressors):
				\begin{center}
					$t = \frac{\hat{\rho}_1}{se(\hat{\rho}_1)} \sim t_{T-k-1, \alpha/2}$
				\end{center}
				\begin{itemize}[leftmargin=*]
					\item $H_1$: Auto-correlation of order one, AR(1).
				\end{itemize}
				\item \textbf{Durbin-Watson statistic} (exogenous regressors and residual normality):
				\begin{center}
					$d = \frac{\sum_{t=2}^n (\hat{u}_t - \hat{u}_{t-1})^2}{\sum_{t=1}^n \hat{u}_t^2} \approx 2 (1 - \hat{\rho}_1)$  ,  $0 \leq d \leq 4$
				\end{center}
				\begin{itemize}[leftmargin=*]
					\item $H_1$: Auto-correlation of order one, AR(1).
				\end{itemize}
				\begin{center}
					\scalebox{0.8}{
						\begin{tabular}{| c | c | c | c |}
							\hline
							$d = $ & 0 & 2 & 4 \\
							\hline
							$\rho \approx$ & -1 & 0 & 1 \\
							\hline
						\end{tabular}
					}
					\begin{tikzpicture}[scale=0.28]
						\node at (-1.5,9.5) {\scalebox{1.2}{\tiny $f(d)$}};
						\draw [thick] (0,10) -- (0,0) -- (25,0); % axis
						\draw [very thin, dashed] (12.5,0) -- (12.5,10);
						\fill [pattern=north west lines, pattern color=black] (5,0)  rectangle  (9,10);
						\draw (5,0) -- (5,10);
						\draw (9,0) -- (9,10);
						\fill [pattern=north west lines, pattern color=black] (16,0) rectangle (20,10);
						\draw (16,0) -- (16,10);
						\draw (20,0) -- (20,10);
						
						\node at (0,-0.6) {\scalebox{1.2}{\tiny 0}};
						\node at (5,-0.6) {\scalebox{1.2}{\tiny $d_L$}};
						\node at (9,-0.6) {\scalebox{1.2}{\tiny $d_U$}};
						\node at (12.5,-0.6) {\scalebox{1.2}{\tiny 2}};
						\node at (16.7,-1) {\scalebox{1.1}{\tiny \rotatebox{-20}{$4 - d_U$}}};
						\node at (20.7,-1) {\scalebox{1.1}{\tiny \rotatebox{-20}{$4 - d_L$}}};
						\node at (25,-0.6) {\scalebox{1.2}{\tiny 4}};
						
						\node at (2.5,5.5) {\scalebox{1.2}{\tiny Rej. $H_0$}};
						\node at (2.5,4.5) {\scalebox{1.2}{\tiny AR(+)}};
						\node [text=red] at (7,5) {\scalebox{1.4}{\tiny \rotatebox{-70}{\textbf{INCONCLUSIVE}}}};
						\node at (12.5,5.5) {\scalebox{1.2}{\tiny Accept $H_0$}};
						\node at (12.5,4.5) {\scalebox{1.2}{\tiny No AR}};
						\node [text=red] at (18,5) {\scalebox{1.4}{\tiny \rotatebox{-70}{\textbf{INCONCLUSIVE}}}};
						\node at (22.5,5.5) {\scalebox{1.2}{\tiny Rej. $H_0$}};
						\node at (22.5,4.5) {\scalebox{1.2}{\tiny AR(-)}};
					\end{tikzpicture}
				\end{center}
				\item \textbf{Durbin's h} (endogenous regressors):
				\begin{center}
					$h = \hat{\rho} \sqrt{\frac{T}{1 - T \times \upsilon}}$
				\end{center}
				where $\upsilon$ is the estimated variance of the coefficient associated to the endogenous variable.
				\begin{itemize}[leftmargin=*]
					\item $H_1$: Auto-correlation of order one, AR(1).
				\end{itemize}
				\item \textbf{Breusch-Godfrey test} (endogenous regressors): it can detect MA($q$) and AR($p$) processes ($\varepsilon_t$ is w. noise):
				\begin{itemize}[leftmargin=*]
					\item MA($q$): $u_t = \varepsilon_t - \theta_1 u_{t-1} - ... - \theta_q u_{t-q}$
					\item AR($p$): $u_t = \rho_1 u_{t-1} + ... + \rho_p u_{t-p} + \varepsilon_t$
				\end{itemize}
				Under $H_0$: No auto-correlation:
				\begin{center}
					$\hfill T \times R^2_{\hat{u}_t} \sim_a \chi^2_q \hfill \textbf{or} \hfill T \times R^2_{\hat{u}_t} \sim_a \chi^2_p \hfill$
				\end{center}
				\begin{itemize}[leftmargin=*]
					\item $H_1$: Auto-correlation of order $q$ (or $p$).
				\end{itemize}
				\item \textbf{Ljung-Box Q test}:
				\begin{itemize}[leftmargin=*]
					\item $H_1$: There is auto-correlation.
				\end{itemize}
			\end{itemize}
		\end{itemize}
	\subsection*{Correction}
		\begin{itemize}[leftmargin=*]
			\item Use OLS with a \textbf{variance-covariance matrix} estimator \textbf{robust to auto-correlation}, for example, the one proposed by \textbf{Newey-West}.
			\item Use \textbf{Generalized Least Squares}. Supposing $y_t = \beta_0 + \beta_1 x_t + u_t$, with $u_t = \rho u_{t-1} + \varepsilon_t$, where $|\rho| < 1$ and $\varepsilon_t$ is white noise.
			\begin{itemize}[leftmargin=*]
				\item If \textbf{$\rho$ is known, create a quasi-differentiated model}:
				\begin{center}
					$y_t - \rho y_{t-1} = \beta_0 (1 - \rho) + \beta_1 (x_t - \rho x_{t-1}) + u_t - \rho u_{t-1}$
					\ $y_t^* = \beta_0^* + \beta_1 x_t^* + u_t^*$
				\end{center}
				where $u_t^*$ is white noise and estimate it by OLS.
				\item If \textbf{$\rho$ is not known, estimate it by} -for example- \textbf{the Cochrane-Orcutt method} (Prais-Winsten method is also good):
				\begin{enumerate}[leftmargin=*]
					\item Obtain $\hat{u}_t$ from the original model.
					\item Estimate $\hat{u}_t = \rho \hat{u}_{t-1} + \varepsilon_t$ and obtain $\hat{\rho}$.
					\item Estimate a quasi-differentiated model:
					\begin{center}
						$y_t - \hat{\rho} y_{t-1} = \beta_0 (1 - \hat{\rho}) + \beta_1 (x_t - \hat{\rho} x_{t-1}) + u_t - \hat{\rho} u_{t-1}$
						\ $y_t^* = \beta_0^* + \beta_1 x_t^* + u_t^*$
					\end{center}
					where $u_t^*$ is white noise and estimate it by OLS.
					\item Obtain $\hat{u}_t^*$ and repeat from step 2.
					\item The method finish when the estimated parameters vary very little between iterations.
				\end{enumerate}
			\end{itemize}
		\item If \textbf{not solved, look for high dependence} in the series.
		\end{itemize}

\section*{Stationarity and weak dependence}
	Stationarity means stability of the joints distributions of a process as time progresses. It allows to correctly identify the relations --that stay unchange with time-- between variables.
	\subsection*{Stationary and non-stationary process}
		\begin{itemize}[leftmargin=*]
			\item \textbf{Stationary process} (strong stationarity) - is the one in that the probability distribution are stable in time: if any collection of random variables is taken, and then are shifted $h$ periods, the joint probability distribution should stay unchanged.

\columnbreak

			\item \textbf{Non-stationary process} - is, for example, a series with trend, where at least the mean changes with the time.
			\item \textbf{Covariance stationary process} - is a weaker form of stationarity:
			\begin{itemize}[leftmargin=*]
				\item $E(x_t)$ is constant.
				\item $Var(x_t)$ is constant.
				\item For any $t$,  $h \geq 1$, the $Cov(x_t, x_{t+h})$ depends only of $h$, not of $t$.
			\end{itemize}
		\end{itemize}
	\subsection*{Weak dependence time series}
		It is important because it replaces the random sampling assumption, giving for granted the validity of the central limit theorem (requires stationality and a form of weak dependence). Weakly dependent process are \textbf{also known as integrated of order zero, I(0)}.
		\begin{itemize}[leftmargin=*]
			\item \textbf{Weak dependence} - restricts how close the relationship between $x_t$ and $x_{t+h}$ can be as the time distance between the series increases ($h$).
		\end{itemize}
		A stationary time process $\lbrace x_t: t = 1, 2, ..., T \rbrace$ is weakly dependent when $x_t$ and $x_{t+h}$ are almost independent as $h$ increases without a limit.
		\\ A covariance stationary time process is weakly dependent if the correlation between $x_t$ and $x_{t+h}$ tends to $0$ fast enough when $h \rightarrow \infty$ (they are not asymptotically correlated).
		\\ Some examples of stationary and weakly dependent time series are:
		\begin{itemize}[leftmargin=*]
			\item \textbf{Moving average} - $\lbrace x_t \rbrace$ is a moving average of order one MA($q=1$):
			\begin{center}
				$x_t = e_t + \theta_1 e_{t-1}$
			\end{center}
			where $\lbrace e_t: t = 0, 1, ..., T \rbrace$ is an \textsl{i.i.d.} sequence with zero mean and $\sigma^2_e$ variance.
			\item \textbf{Auto-regressive process} - $\lbrace x_t \rbrace$ is an auto-regressive process of order one AR($p=1$):
			\begin{center}
				$x_t = \rho_1 x_{t-1} + e_t$
			\end{center}
			where $\lbrace e_t: t = 0, 1, ..., T \rbrace$ is an \textsl{i.i.d.} sequence with zero mean and $\sigma^2_e$ variance.
			\\ If $|\rho_1|<1$, then $\lbrace x_t \rbrace$ is an AR(1) stable process that is weakly dependent. It is stationary in covariance, $Corr(x_t, x_{t-1}) = \rho_1$.
			\item \textbf{ARMA process} - is a combination of the two above. $\lbrace x_t \rbrace$ is an ARMA($p=1$,$q=1$):
			\begin{center}
				$x_t = e_t + \theta_1 e_{t-1} + \rho_1 x_{t-1}$
			\end{center}
		\end{itemize}
		A series with a trend cannot be stationary, but can be weakly dependent (and stationary if the series is de-trended).

\columnbreak

\section*{Strong dependence time series}
	Most of the time, economics series are strong dependent (or high persistent in time). Some special cases of unit root processes:
	\begin{itemize}[leftmargin=*]
		\item \textbf{Random walk} - an AR(1) process with $\rho_1 = 1$.
		\begin{center}
			$y_t = y_{t-1} + e_t$
		\end{center}
		where $\lbrace e_t : t = 1, 2, ..., T \rbrace$ is an \textsl{i.i.d.} sequence with zero mean and $\sigma^2_e$ variance (the latter changes with time).
		\\ This is a random walk ($\lbrace e_t \rbrace$ \textsl{i.i.d.} is the reason). It is not stationary, is persistent.
		\item \textbf{Random walk with a \href{https://www.youtube.com/watch?v=pS5d77DQHOI}{drift}} - an AR(1) process with $\rho_1 = 1$ and a constant.
		\begin{center}
			$y_t = \alpha + y_{t-1} + e_t$
		\end{center}
		where $\lbrace e_t : t = 1, 2, ..., T \rbrace$ is an \textsl{i.i.d.} sequence with zero mean and $\sigma^2_e$ variance.
		\\ This is a random walk ($\lbrace e_t \rbrace$ \textsl{i.i.d.} is the reason). It is not stationary, is persistent.
	\end{itemize}
	\subsection*{I(1) detection}
		\begin{itemize}[leftmargin=*]
			\item \textbf{Augmented Dickey-Fuller (ADF) test} - where $H_0$: the process is unit root, I(1).
			\item \textbf{Kwiatkowski–Phillips–Schmidt–Shin (KPSS) test} - where $H_0$: the process have no unit root, I(0).
		\end{itemize}
	\subsection*{Transforming unit root to weak dependent}
		Unit root processes are \textbf{integrated of order one}, I(1). This means that \textbf{the first difference of the process is weakly dependent} or I(0) (and usually, stationary). For example, a random walk:
		\begin{multicols}{2} % set columns to 2
			\begin{center}
				$\Delta y_t = y_t - y_{t-1} = e_t$
			\end{center}
			where $\lbrace e_t \rbrace = \lbrace \Delta y_t \rbrace$  is \textsl{i.i.d.}
			\\ 
			\\ Getting the first difference of a series also deletes its trend.
			\\ For example, a series with a trend (black), and it's first difference ({\color{red} red}).
			\columnbreak
			\begin{tikzpicture}[scale=0.18]
				\draw [step=2, gray, very thin] (-10,-10) grid (10,10); % background grid
				\draw [thick, <->] (-10,10) node [anchor=south] {$y$} -- (-10,-10) -- (10,-10) node [anchor=north] {$t$}; %axis
				\draw (-8,-8) -- 
				(-7.6,-7.5411) -- 
				(-7.2,-7.2841) -- 
				(-6.8,-6.7952) -- 
				(-6.4,-6.4292) -- 
				(-6,-6.0486) -- 
				(-5.6,-5.9534) -- 
				(-5.2,-5.4861) -- 
				(-4.8,-5.2817) -- 
				(-4.4,-4.8405) -- 
				(-4,-4.3267) -- 
				(-3.6,-4.0131) -- 
				(-3.2,-3.7582) -- 
				(-2.8,-3.5296) -- 
				(-2.4,-3.0564) -- 
				(-2,-2.8965) -- 
				(-1.6,-2.4169) -- 
				(-1.2,-1.9134) -- 
				(-0.8,-1.8881) -- 
				(-0.4,-1.1661) -- 
				(0,-0.53) -- 
				(0.4,-0.2821) -- 
				(0.8,0.0327) -- 
				(1.2,0.4915) -- 
				(1.6,0.7483) -- 
				(2,0.8053) -- 
				(2.4,1.0167) -- 
				(2.8,1.4391) -- 
				(3.2,1.8107) -- 
				(3.6,2.2473) -- 
				(4,2.6688) -- 
				(4.4,3.052) -- 
				(4.8,3.5867) -- 
				(5.2,4.322) -- 
				(5.6,4.9132) -- 
				(6,5.7041) -- 
				(6.4,6.0819) -- 
				(6.8,6.4316) -- 
				(7.2,6.8585) -- 
				(7.6,7.5108) -- 
				(8,8);
				\draw [red] (-7.6,1.2835) -- 
				(-7.2,-2.7995) -- 
				(-6.8,1.8898) -- 
				(-6.4,-0.5953) -- 
				(-6,-0.2992) -- 
				(-5.6,-6.0745) -- 
				(-5.2,1.4544) -- 
				(-4.8,-3.8646) -- 
				(-4.4,0.9266) -- 
				(-4,2.3932) -- 
				(-3.6,-1.6553) -- 
				(-3.2,-2.8432) -- 
				(-2.8,-3.3737) -- 
				(-2.4,1.5724) -- 
				(-2,-4.7658) -- 
				(-1.6,1.7033) -- 
				(-1.2,2.1863) -- 
				(-0.8,-7.4877) -- 
				(-0.4,6.6075) -- 
				(0,4.869) -- 
				(0.4,-2.9853) -- 
				(0.8,-1.6322) -- 
				(1.2,1.2832) -- 
				(1.6,-2.8046) -- 
				(2,-6.8477) -- 
				(2.4,-3.7232) -- 
				(2.8,0.547) -- 
				(3.2,-0.4838) -- 
				(3.6,0.8346) -- 
				(4,0.5268) -- 
				(4.4,-0.2468) -- 
				(4.8,2.816) -- 
				(5.2,6.8759) -- 
				(5.6,3.9619) -- 
				(6,8) -- 
				(6.4,-0.3568) -- 
				(6.8,-0.9251) -- 
				(7.2,0.6366) -- 
				(7.6,5.1971) -- 
				(8,1.8972);
			\end{tikzpicture}
		\end{multicols}
		When the I(1) series is strictly positive, usually it is converted to logarithms before taking the first difference. That is, to obtain the (approx.) percentage change of the series, by taking the first difference of a logarithm:
		\begin{center}
			$\Delta log(y_t) = log(y_t) - log(y_{t-1}) \approx (y_t - y_{t-1}) / y_{t-1}$
		\end{center}

\columnbreak

\section*{Cointegration}
	When \textbf{two series are I(1), but a linear combination of them is I(0)}. If the case, the regression of one series over the other is not spurious, but expresses something about the long term relation.
	\\ For example: $\lbrace x_t \rbrace$ and $\lbrace y_t \rbrace$ are I(1), but $y_t - \beta x_t = u_t$ where $\lbrace u_t \rbrace$ is I(0). ($\beta$ get the name of cointegration parameter).

\section*{Heterocedasticity on time series}
	The assumption \textbf{affected is ts4}, which leads to \textbf{OLS be not efficient}. 
	\\ Some tests that work could be the Breusch-Pagan or White's, where $H_0$: No heterocedasticity. It is \textbf{important for the tests to work that there is no auto-correlation} (so first, it is important to test for auto-correlation).
	\subsection*{ARCH}
		An auto-regressive conditional heterocedasticity (ARCH), is a model to analyze a form of dynamic heterocedasticity, where the error variance follows an AR($p$) process.
		\\ Given the model:
		\begin{center}
			$y_t = \beta_0 + \beta_1 z_t + u_t$
		\end{center}
		where, there is AR(1) and heterocedasticity:
		\begin{center}
			$E(u^2_t | u_{t-1}) = \alpha_0 + \alpha_1 u^2_{t-1}$
		\end{center}
	\subsection*{GARCH}
		A general auto-regressive conditional heterocedasticity (GARCH), is a model similar to ARCH, but in this case, the error variance follows an ARMA($p$,$q$) process.

\section*{Predictions}
	Two types of prediction:
	\begin{itemize}[leftmargin=*]
		\item Of the mean value of $y$ for a specific value of $x$.
		\item Of an individual value of $y$ for a specific value of $x$.
	\end{itemize}
	If the values of the variables ($x$) approximate to the mean values ($\overline{x}$), the confidence interval amplitude of the prediction will be shorter. 

\end{multicols}

\end{document}