\documentclass[10pt, a4paper, landscape]{extarticle}

% -----packages-----
\usepackage{multicol} % for multiple columns
\usepackage[landscape]{geometry} % for landscape
\usepackage{parskip} % remove text indentation
\usepackage{graphicx} % for scale tables
\usepackage{enumitem} % indent of lists
\usepackage{tikz} % for plots
\usepackage{hyperref} % for hyperlinks
\usepackage{amsmath} % for writing normal text on equations
\usepackage{scrlayer-scrpage} % page foot
\usepackage[compact]{titlesec} % titles spacing

% -----page customization-----
\geometry{top=1cm,left=1cm,right=1cm,bottom=1cm} % margins configuration
\pagenumbering{gobble} % remove page numeration
\setlength{\parskip}{0cm} % paragraph skip length
% title spacing:
\titlespacing{\section}{0pt}{2ex}{1ex}
\titlespacing{\subsection}{0pt}{1ex}{0ex}
\titlespacing{\subsubsection}{0pt}{0.5ex}{0ex}

% -----document-----
\begin{document}

\cfoot{\href{https://github.com/marcelomijas/econometrics-cheatsheet}{\normalfont \footnotesize Version ts0.7-en - github.com/marcelomijas/econometrics-cheatsheet}}
\setlength{\footskip}{12pt}

\begin{multicols}{3} % set columns to 3

\begin{center}
	\textbf{\LARGE \href{https://github.com/marcelomijas/econometrics-cheatsheet}{Time Series Cheat Sheet}}
	\\ {\footnotesize By Marcelo Moreno - King Juan Carlos University} 
	\\ {\footnotesize As part of the Econometrics Cheat Sheet Project}
\end{center}

\colorbox{yellow}{THIS IS A WORK IN PROGRESS}
\\ \colorbox{yellow}{NOT INTENDED FOR GENERAL PURPOSE YET}

\section*{Basic concepts}
	\subsection*{Definitions}
		\textbf{Time series} - is a succession of quantitative observations of a phenomena ordered in time.
		\\ There are some variations of time series:
		\begin{itemize}[leftmargin=*]
			\item \textbf{Panel data} - consist of a time series for each observation of a cross section.
			\item \textbf{Pooled cross sections} - combines cross sections from different time periods.
		\end{itemize}
		\textbf{Stochastic process} - sequence of random variables that are indexed in time.
	\subsection*{Components of a time series}
		\begin{itemize}[leftmargin=*]
			\item \textbf{Trend} - is the long-term general movement of a series.
			\item \textbf{Seasonal variations} - are periodic oscillations that are produced in a period equal or inferior to a year, and can be easily identified on different years (usually are the result of climatology reasons).
			\item \textbf{Cyclical variations} - are periodic oscillations that are produced in a period greater than a year (are the result of the economic cycle).
			\item \textbf{Residual variations} - are movements that do not follow a recognizable periodic oscillation (are the result of eventual non-permanent phenomena that can affect the studied variable in a given moment).
		\end{itemize}

	\subsection*{Type of time series models}
		\begin{itemize}[leftmargin=*]
			\item \textbf{Static models} - the relation between $y$ and $x$'s is contemporary. Conceptually:
			\begin{center}
				$y_t = \beta_0 + \beta_1 x_t + u_t$
			\end{center}
			\item \textbf{Distributed-lag models} - the relation between $y$ and $x$'s is not contemporary. Conceptually:
			\begin{center}
				$y_t = \beta_0 + \beta_1 x_t + \beta_2 x_{t-1} + ... + \beta_{q+1} x_{t-q} + u_t$
			\end{center}
			The long term cumulative effect in $y$ when $\Delta x$ is:
			 \begin{center}
			 	$\beta_1 + \beta_2 + ... + \beta_{q+1}$
			 \end{center}
		\end{itemize}

\columnbreak

\section*{Assumptions and properties}
	\subsection*{OLS model assumptions under time series}
		Under this assumptions, the estimators of the OLS parameters will present good properties. \textbf{Gauss-Markov assumptions extended applied to time series}:
		\begin{enumerate}[leftmargin=*, label=ts\arabic*.]
			\item \textbf{Parameters linearity and weak dependence}.
			\begin{enumerate}[leftmargin=*, label=\alph*.]
				\item $y_t$ must be a linear function of the $\beta$'s.
				\item The stochastic $\lbrace(x_t, y_t): t = 1, 2, ..., T\rbrace$ is stationary and weakly dependent.
			\end{enumerate} 
			\item \textbf{No perfect collinearity}.
			\begin{itemize}[leftmargin=*]
				\item There are no independent variables that are constant: $Var(x_j) \neq 0$
				\item There is not an exact linear relation between independent variables.
			\end{itemize}
			\item \textbf{Conditional mean zero and correlation zero}.
			\begin{enumerate}[leftmargin=*, label=\alph*.]
				\item There are no systematic errors: $E(u_t | x_{1t}, ..., x_{kt}) = E(u_t) = 0 \rightarrow$ \textbf{strong exogeneity} (a implies b).
				\item There are no relevant variables left out of the model: $Cov(x_{jt} , u_t) = 0$ for any $j = 1, ..., k \rightarrow$ \textbf{weak exogeneity}.
			\end{enumerate}
			\item \textbf{Homoscedasticity}. The variability of the residuals is the same for any $t$: $Var(u_t | x_{1t}, ..., x_{kt}) = \sigma^2$
			\item \textbf{No auto-correlation}. The residuals do not contain information about other residuals: $Corr(u_t, u_s | x) = 0$ for any given $t \neq s$.
			\item \textbf{Normality}. The residuals are independent and identically distributed (\textbf{i.i.d.} so on): $u \sim N(0,\sigma^2)$
			\item \textbf{Data size}. The number of observations available must be greater than $(k + 1)$ parameters to estimate. (IS already satisfied under asymptotic situations)
		\end{enumerate}	
	\subsection*{Asymptotic properties of OLS}
		Under the econometric model assumptions and the Central Limit Theorem:
		\begin{itemize}[leftmargin=*]
			\item Hold (1) to (3a): OLS is \textbf{unbiased}. $E(\hat{\beta}_j) = \beta_j$
			\item Hold (1) to (3): OLS is \textbf{consistent}. $plim(\hat{\beta}_j) = \beta_j$ (to (3b) left out (3a), weak exogeneity, biased but consistent)
			\item Hold (1) to (5): \textbf{asymptotic normality} of OLS (then, (6) is necessarily satisfied): $u \sim_a N(0,\sigma^2)$.
			\item Hold (1) to (5): \textbf{unbiased estimate of $\sigma^2$}. $E(\hat{\sigma}^2) = \sigma^2$
			\item Hold (1) to (5): OLS is \textcolor{blue}{BLUE} (Best Linear Unbiased Estimator) or \textbf{efficient}. 
			\item Hold (1) to (6): hypothesis testing and confidence intervals can be done reliably.
		\end{itemize}

\section*{Trends and seasonality}
	\textbf{Spurious regression} - is when the relation between $y$ and $x$ is due to factors that affect $y$ and have correlation with $x$, $Corr(x, u) \neq 0$. 
	\subsection*{Trends}
		Two time series can have the same (or contrary) trend, that should lend to a high level of correlation. This, can provoke a false appearance of causality, the problem is in what is known as \textbf{spurious regression} (the non-fulfillment of ts3). For example, given the model:
		\begin{center}
			$y_t = \beta_0 + \beta_1 x_t + u_t$
		\end{center}
		where:
		\begin{center}
			$y_t = \alpha_0 + \alpha_1 Trend + v_t$
			\\ $x_t = \gamma_0 + \gamma_1 Trend + v_t$
		\end{center}
		Adding a trend to the model can solve the problem:
		\begin{center}
			$y_t = \beta_0 + \beta_1 x_t + \beta_2 Trend + u_t$
		\end{center}
		The trend can be linear or non-linear (quadratic, cubic, exponential, etc.)
	\subsection*{Seasionality}
		\setlength{\multicolsep}{0pt} % reduce vertical spacing betwen subsection and multicols
		\begin{multicols}{2} % set columns to 2
		A time series with can manifest seasionality. That is, the series is subject to a seasonal variations or pattern, usually related to climatology conditions.
		\\ For example, the GDP is usually higher in summer and lower in winter.
		\columnbreak
			\begin{tikzpicture}[scale=0.18]
				\draw[step=2, gray, very thin] (-10,-10) grid (10,10); % background grid
				\draw[thick, <->] (-10,10) node[anchor=south] {$y$} -- (-10,-10) -- (10,-10) node[anchor=north] {$t$}; %axis
				\draw[red, thick] plot [domain=-10:10] (\x,{-0.19 + 0.66 * \x}); % regression line
				\draw (-8,-7.2061) -- (-7.5676,-5.1908) -- (-7.1351,-8) -- (-6.7027,-2.3817) -- (-6.2703,-3.9695) -- (-5.8378,-1.1603) -- (-5.4054,-4.5802) -- (-4.973,0.855) -- (-4.5405,-1.5267) -- (-4.1081,-0.3053) -- (-3.6757,-4.5191) -- (-3.2432,-0.4885) -- (-2.8108,-2.3206) -- (-2.3784,-0.4275) -- (-1.9459,-4.2137) -- (-1.5135,0.3664) -- (-1.0811,-1.7099) -- (-0.6486,-0.5496) -- (-0.2162,-4.3969) -- (0.2162,1.0992) -- (0.6486,-1.0382) -- (1.0811,1.2824) -- (1.5135,-2.8702) -- (1.9459,1.7099) -- (2.3784,-1.0382) -- (2.8108,1.5267) -- (3.2432,-1.8321) -- (3.6757,3.3588) -- (4.1081,1.0992) -- (4.5405,4.2137) -- (4.973,0.916) -- (5.4054,6.2901) -- (5.8378,4.3969) -- (6.2703,6.5954) -- (6.7027,3.4198) -- (7.1351,8) -- (7.5676,6.1069) -- (8,6.9008);
			\end{tikzpicture}
		\end{multicols}
		This problem is part of what is known as \textbf{spurious regression} (the non-fulfillment of ts3). A seasonal adjustment can solve the problem.
		\\ A simple seasonal adjustment could be creating stational binary variables and adding them to the model. For example, for quarterly series:
		\begin{center}
			$y_t = \beta_0 + \beta_1 Q2_t + \beta_2 Q3_t + \beta_3 Q4_t + \beta_4 x_{1t} + ... + \beta_k x_{kt} + u_t$
		\end{center}
		Another way is to seasonaly adjust (sa) the variables, and then, do the regression with the final regression:
		\begin{center}
			$z_t = \beta_0 + \beta_1 Q2_t + \beta_2 Q3_t + \beta_3 Q4_t  + v_t \rightarrow (sa)\tilde{v}_t = (sa)\tilde{z}_t$
			$(sa)\tilde{y}_t = \beta_0 + \beta_1 (sa)\tilde{x}_{1t} + ... + \beta_k (sa)\tilde{x}_{kt} + u_t$
		\end{center}
		There are much better and complex methods to seasonaly adjust a time series.
		
\columnbreak

\section*{Stationarity and weak dependence}
	Stationarity means stability of the joints distributions of a process as time progresses. It allows to correctly identify the relations --that stay unchange with time-- between variables.
	\subsection*{Stationary and non-stationary process}
		\begin{itemize}[leftmargin=*]
			\item \textbf{Stationary process} (strong stationarity) - is the one in that the probability distribution are stable in time: if any collection of random variables is taken, and then are shifted $h$ periods, the joint probability distribution should stay unchanged.
			\item \textbf{Non-stationary process} - is, for example, a series with trend, where at least the mean changes with the time.
			\item \textbf{Covariance stationary process} - is a weaker form of stationarity:
			\begin{itemize}[leftmargin=*]
			 	\item $E(x_t)$ is constant.
			 	\item $Var(x_t)$ is constant.
			 	\item For any $t$,  $h \geq 1$, the $Cov(x_t, x_{t+h})$ depends only of $h$, not of $t$.
			\end{itemize}
		\end{itemize}
	\subsection*{Weak dependence time series}
		It is important because it replaces the random sampling assumption, giving for granted the validity of the central limit theorem (requires stationality and a form of weak dependence). Weakly dependent process are \textbf{also known as integrated of order zero, I(0)}.
		\begin{itemize}[leftmargin=*]
			\item \textbf{Weak dependence} - restricts how close the relationship between $x_t$ and $x_{t+h}$ can be as the time distance between the series increases ($h$).
		\end{itemize}
		A stationary time process $\lbrace x_t: t = 1, 2, ..., T \rbrace$ is weakly dependent when $x_t$ and $x_{t+h}$ are almost independent as $h$ increases without a limit.
		\\ A covariance stationary time process is weakly dependent if the correlation between $x_t$ and $x_{t+h}$ tends to $0$ fast enough when $h \rightarrow \infty$ (they are not asymptotically correlated).
		\\ Some examples of stationary and weakly dependent time series are:
		\begin{itemize}[leftmargin=*]
			\item \textbf{Moving average} - $\lbrace x_t \rbrace$ is a moving average of order one MA(1).
			\begin{center}
				$x_t = e_t + \alpha_1 e_{t-1}$
			\end{center}
			where $\lbrace e_t: t = 0, 1, ..., T \rbrace$ is an \textsl{i.i.d.} sequence with zero mean and $\sigma^2_e$ variance.
			\item \textbf{Auto-regressive process} - $\lbrace y_t \rbrace$ is a auto-regressive process of order one AR(1).
			\begin{center}
				$y_t = \rho_1 y_{t-1} + e_t$
			\end{center}
			where $\lbrace e_t: t = 0, 1, ..., T \rbrace$ is an \textsl{i.i.d.} sequence with zero mean and $\sigma^2_e$ variance.
		\\ If $|\rho_1|<1$, then $\lbrace y_t \rbrace$ is an AR(1) stable process that is weakly dependent. It is stationary in covariance, $Corr(y_t, y_{t+1}) = \rho_1$.
		\end{itemize}
	A series with a trend cannot be stationary, but can be weakly dependent (and stationary if the series is de-trended).
	
	\subsection*{Strong dependence time series}
		Most of the time, economics series are strong dependent (or high persistent in time). Some special cases of unit root processes:
		\begin{itemize}[leftmargin=*]
			\item \textbf{Random walk} - an AR(1) process with $\rho_1 = 1$.
			\begin{center}
				$y_t = y_{t-1} + e_t$
			\end{center}
			where $\lbrace e_t : t = 1, 2, ..., T \rbrace$ is an \textsl{i.i.d.} sequence with zero mean and $\sigma^2_e$ variance.
			\\ This is a random walk ($\lbrace e_t \rbrace$ \textsl{i.i.d.} is the reason). It is not stationary, is persistent.
			\item \textbf{Random walk with a drift} - an AR(1) process with $\rho_1 = 1$ and a constant.
			\begin{center}
				$y_t = \alpha_0 + y_{t-1} + e_t$
			\end{center}
			where $\lbrace e_t : t = 1, 2, ..., T \rbrace$ is an \textsl{i.i.d.} sequence with zero mean and $\sigma^2_e$ variance.
			\\ This is a random walk ($\lbrace e_t \rbrace$ \textsl{i.i.d.} is the reason). It is not stationary, is persistent.
		\end{itemize}

	\subsection*{Transforming unit root to weak dependent}
		Unit root processes are integrated of order one, I(1). This means that the first difference of the process is weakly dependent (and usually, stationary). For example, a random walk:
		\begin{center}
			$\Delta y_t = y_t - y_{t-1} = e_t$
		\end{center}
		where $\lbrace e_t \rbrace = \lbrace \Delta y_t \rbrace$  is \textsl{i.i.d.}
		\\ Usually, I(1) series are converted to logarithms in order to, getting the first difference to achieve weak dependence (and possibly stationarity), and obtain the (approx.) percentage change:
		\begin{center}
			$\Delta log(y_t) = log(y_t) - log(y_{t-1}) \approx (y_t - y_{t-1}) / y_{t-1}$
		\end{center}
		Getting the first difference of a series also deletes its trend.
		[GRÁFICO]
		
	\subsection*{I(1) detection}
		\begin{itemize}[leftmargin=*]
			\item The approx. way: in a AR(1) model (if the series have a trend, is better to de-trend before measuring the AC). In a AR(1) model:
			\begin{itemize}[leftmargin=*]
				\item If $|\rho_1| < 1 \rightarrow I(0)$
				\item If $|\rho_1| = 1 \rightarrow I(1)$
				\item If $|\rho_1| > 1 \rightarrow try AR(2) and test$
			\end{itemize}
		\end{itemize}
		

\columnbreak

\section*{Predictions}
	Two types of prediction:
	\begin{itemize}[leftmargin=*]
		\item Of the mean value of $y$ for a specific value of $x$.
		\item Of an individual value of $y$ for a specific value of $x$.
	\end{itemize}
	If the values of the variables ($x$) approximate to the mean values ($\overline{x}$), the confidence interval amplitude of the prediction will be shorter. 

\section*{Heteroscedasticity in time series}
	The residuals $u_i$ of the population regression function do not have the same variance $\sigma^2$:
	\begin{center}
		$Var(u|x) = Var(y|x) \neq \sigma^2$
	\end{center}
	Is the \textbf{breaking of the fifth (5) econometric model assumption}.
	\subsection*{Consequences}
		\begin{itemize}[leftmargin=*]
			\item OLS estimators still are unbiased.
			\item OLS estimators still are consistent.
			\item OLS is \textbf{not efficient} anymore, but still a LUE (Linear Unbiased Estimator).
			\item \textbf{Variance estimations of the estimators are biased}: the construction of confidence intervals and the hypothesis contrast are not reliable.
		\end{itemize}
	\subsection*{Detection}
		\begin{itemize}[leftmargin=*]
			\setlength{\multicolsep}{0pt} % reduce vertical spacing betwen subsection and multicols
			\setlength{\columnsep}{20pt} % increment spacing between columns
			\begin{multicols}{3} % set columns to 3
			\item \textbf{Graphical analysis} - look for scatter patterns on $x$ vs. $u$ or $x$ vs. $y$ plots.

			\columnbreak
			\vspace*{-23pt}

			\begin{tikzpicture}[scale=0.108]
				\draw[step=2, gray, very thin] (-10,-10) grid (10,10); % grid
				\draw[thick,->] (-10,0) -- (10,0) node[anchor=north] {$x$}; % x axis
				\draw[thick,-] (-10,-10) -- (-10,10) node[anchor=west] {$u$}; % u axis
				\draw plot [domain=0:17, samples=50] ({\x - 9},{-0.5 * rand * \x - 1}); % data points
				\draw[thick, dashed, red, -latex] plot [domain=1:17] ({\x - 10},{-0.5 * \x - 1}); % lower red arrow
				\draw[thick, dashed, red, -latex] plot [domain=1:17] ({\x - 10},{0.5 * \x - 1}); % upper red arrow
			\end{tikzpicture}

			\columnbreak
			\vspace*{-24pt}

			\begin{tikzpicture}[scale=0.108]
				\draw[step=2, gray, very thin] (-10,-10) grid (10,10); % grid
				\draw[thick,<->] (-10,10) node[anchor=west] {$y$} -- (-10,-10) -- (10,-10) node[anchor=south] {$x$}; % axis
				\draw plot [domain=0:13, samples=50] ({\x -9},{(-0.65 * rand * \x) + 0.6 * \x - 8}); % data points
				\draw[thick, dashed, red, -latex] plot [domain=0:12] ({\x - 9},{-0.06 * \x - 8.5}); % lower red arrow
				\draw[thick, dashed, red, -latex] plot [domain=0:12] ({\x -9},{1.25 * \x - 7.2}); % upper red arrow
			\end{tikzpicture}
			\end{multicols}
			\item \textbf{Formal tests} - White, Bartlett, Breusch-Pagan, etc. Commonly, the null hypothesis: $H_0 = \text{Homoscedasticity}$
		\end{itemize}

	\subsection*{Correction}
		\begin{itemize}[leftmargin=*]
			\item Use OLS with a variance-covariance matrix estimator robust to heteroscedasticity, for example, the one proposed by White.
			\item If the variance structure is known, make use of Weighted Least Squares (WLS) or Generalized Least Squares (GLS).
			\item If the variance structure is not known, make use of Feasible Weighted Least Squared (FWLS), that estimates a possible variance, divides the model variables by it and then apply OLS.
			\item Make assumptions about the possible variance:
			\begin{itemize}[leftmargin=*]
				\item Supposing that $\sigma_i^2$ is proportional to $x_i$, divide the model variables by the square root of $x_i$ and apply OLS.
				\item Supposing that $\sigma_i^2$ is proportional to $x_i^2$, divide the model variables by $x_i$ and apply OLS.
			\end{itemize}
			\item Make a new model specification, for example, logarithmic transformation.
		\end{itemize}

\section*{Auto-correlation}
	The residual of any observation, $u_t$, is correlated with the residual of any other observation. The observations are not independent.
	\begin{center}
		$Corr(u_t, u_s | x) \neq 0$ for any $t \neq s$
	\end{center}
	The ``natural" context of this phenomena is time series. Is the \textbf{breaking of the sixth (6) econometric model assumption}.
	\subsection*{Consequences}
		\begin{itemize}[leftmargin=*]
			\item OLS estimators still are unbiased.
			\item OLS estimators still are consistent.
			\item OLS is \textbf{not efficient} anymore, but still a LUE (Linear Unbiased Estimator).
			\item \textbf{Variance estimations of the estimators are biased}: the construction of confidence intervals and the hypothesis contrast are not reliable.
		\end{itemize}
	\subsection*{Detection}
		\begin{itemize}[leftmargin=*]
			\item \textbf{Graphical analysis} - look for scatter patterns on $u_{t-1}$ vs. $u_t$ or make use of a correlogram.
			\setlength{\multicolsep}{0pt} % reduce vertical spacing betwen subsection and multicols
			\setlength{\columnsep}{6pt} % increment spacing between columns
			\begin{multicols}{3} % set columns to 3
				\begin{center}
					\textbf{\footnotesize AR}
				\end{center}
				\vspace{2.0pt}
				\begin{tikzpicture}[scale=0.11]
					\draw[step=2, gray, very thin] (-10,-10) grid (10,10); % grid
					\draw[thick,->] (-10,0) -- (10,0) node[anchor=south] {$u_{t-1}$}; % ut-1 axis
					\draw[thick,-] (-10,-10) -- (-10,10) node[anchor=west] {$u_t$}; % ut axis
					\draw plot [only marks, mark=*, mark size=6, domain=-8:8, samples=50] (\x,{rnd * 6 + (-2 * (\x)^2 + 40) * 0.1}); % data points
					\draw[thick, dashed, red, -latex] plot [domain=-8:8] (\x,{3 + (-2 * (\x)^2 + 40) * 0.1}); % red arrow
				\end{tikzpicture}

				\columnbreak

				\begin{center}
					\textbf{\footnotesize AR(+)}
				\end{center}
				\begin{tikzpicture}[scale=0.11]
					\draw[step=2, gray, very thin] (-10,-10) grid (10,10); % grid
					\draw[thick,->] (-10,0) -- (10,0) node[anchor=north] {$ u_{t-1}$}; % ut-1 axis
					\draw[thick,-] (-10,-10) -- (-10,10) node[anchor=west] {$u_t$}; % ut axis
					\draw plot [only marks, mark=*, mark size=6, domain=-8:8, samples=25] (\x,{rnd * 6 + 0.5 * \x - 3}); % data points
					\draw[thick, dashed, red, -latex] plot [domain=-8:8] (\x,{3 + 0.5 * \x - 3}); % red arrow
				\end{tikzpicture}

				\columnbreak

				\begin{center}
					\textbf{\footnotesize AR(-)}
				\end{center}
				\begin{tikzpicture}[scale=0.11]
					\draw[step=2, gray, very thin] (-10,-10) grid (10,10); % grid
					\draw[thick,->] (-10,0) -- (10,0) node[anchor=south] {$u_{t-1}$}; % ut-1 axis
					\draw[thick,-] (-10,-10) -- (-10,10) node[anchor=west] {$u_t$}; % ut axis
					\draw plot [only marks, mark=*, mark size=6, domain=-8:8, samples=25] (\x,{rnd * 6 - 0.5 * \x - 3}); % data points
					\draw[thick, dashed, red, -latex] plot [domain=-8:8] (\x,{3 - 0.5 * \x - 3}); % red arrow
				\end{tikzpicture}
			\end{multicols}
			\item \textbf{Formal tests} - Durbin-Watson, Breusch-Godfrey, etc. Commonly, the null hypothesis: $H_0: \text{No auto-correlation}$
			\textbf{DURBIN WATSON EXPLANATION AND GRAPHICAL ANALYSIS (AND MOST COMMON VALUES?)}
		\end{itemize}
	\subsection*{Correction}
		\begin{itemize}[leftmargin=*]
			\item Use OLS with a variance-covariance matrix estimator robust to auto-correlation, for example, the one proposed by Newey-West.
			\item Use Generalized Least Squares. Supposing $y_t = \beta_0 + \beta_1 x_t + u_t$, with $u_t = \rho u_{t-1} + \varepsilon_t$, where $|\rho| < 1$ and $\varepsilon_t$ is white noise.
			\begin{itemize}[leftmargin=*]
				\item If $\rho$ is known, create a quasi-differentiated model where $u_t$ is white noise and estimate it by OLS.
				\item If $\rho$ is not known, estimate it by -for example- the Cochrane-Orcutt method, create a quasi-differentiated model where $u_t$ is white noise and estimate it by OLS.
			\end{itemize}
		\end{itemize}

\end{multicols}

\end{document}