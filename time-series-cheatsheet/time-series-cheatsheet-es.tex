\documentclass[10pt, a4paper, landscape]{extarticle}

% -----packages-----
\usepackage{multicol} % for multiple columns
\usepackage[landscape]{geometry} % for landscape
\usepackage{parskip} % remove text indentation
\usepackage{graphicx} % for scale tables
\usepackage{enumitem} % indent of lists
\usepackage{tikz} % for plots
\usetikzlibrary{patterns} % for patterns
\usepackage{hyperref} % for hyperlinks
\usepackage{amsmath} % for writing normal text on equations
\usepackage{scrlayer-scrpage} % page foot
\usepackage[compact]{titlesec} % titles spacing

% -----page customization-----
\geometry{top=1cm,left=1cm,right=1cm,bottom=1cm} % margins configuration
\pagenumbering{gobble} % remove page numeration
\setlength{\parskip}{0cm} % paragraph skip length
% title spacing:
\titlespacing{\section}{0pt}{2ex}{1ex}
\titlespacing{\subsection}{0pt}{1ex}{0ex}
\titlespacing{\subsubsection}{0pt}{0.5ex}{0ex}

% -----document-----
\begin{document}

\cfoot{\href{https://github.com/marcelomijas/econometrics-cheatsheet}{\normalfont \footnotesize Versión ts1.1-es - github.com/marcelomijas/econometrics-cheatsheet - CC BY 4.0}}
\setlength{\footskip}{12pt}

\begin{multicols}{3} % set columns to 3
% start of column 1 page 1
% titles
\begin{center}
	\textbf{\LARGE \href{https://github.com/marcelomijas/econometrics-cheatsheet}{Cheat Sheet Series de Tiempo}}
	\\ {\footnotesize Por Marcelo Moreno - Universidad Rey Juan Carlos} 
	\\ {\footnotesize Como parte del Econometrics Cheat Sheet Project}
\end{center}
% start of content
\section*{Conceptos básicos}
	\subsection*{Definiciones}
		\textbf{Serie temporal} -  es una sucesión de observaciones cuantitativas de un fenómeno ordenadas en el tiempo.
		\\ Hay algunas variaciones de serie temporal:
		\begin{itemize}[leftmargin=*]
			\item \textbf{Datos de panel} - consiste en una serie temporal para cada observación de una sección cruzada.
			\item \textbf{Secciones transversales agrupadas} - combina secciones cruzadas de diferentes periodos de tiempo.
		\end{itemize}
		\textbf{Proceso estocástico} - es una secuencia de variables aleatorias que están indexadas en el tiempo.
	\subsection*{Componentes de una serie temporal}
		\begin{itemize}[leftmargin=*]
			\item \textbf{Tendencia} - es el movimiento general a l/p de una serie.
			\item \textbf{Variaciones estacionales} - son oscilaciones periódicas que son producidas en un período igual o inferior al año, y pueden ser fácilmente identificadas en diferentes años (usualmente son el resultado de la climatología).
			\item \textbf{Ciclo} - son oscilaciones periódicas que se producen en un periodo mayor al año (son resultado del ciclo económico).
			\item \textbf{Variaciones residuales} - son movimientos que no siguen una oscilación periódica identificable (resultado de fenómenos eventuales no permanentes que pueden afectar a la variable estudiada en un momento dado).
		\end{itemize}
	\subsection*{Tipos de modelos de series temporales}
		\begin{itemize}[leftmargin=*]
			\item \textbf{Modelos estáticos} - la relación entre $y$ y $x$'s es contemporánea. Conceptualmente:
			\begin{center}
				$y_t = \beta_0 + \beta_1 x_t + u_t$
			\end{center}
			\item \textbf{Modelos de rezagos distribuidos} - la relación entre $y$ y $x$'s no es contemporánea. Conceptualmente:
			\begin{center}
				$y_t = \beta_0 + \beta_1 x_t + \beta_2 x_{t-1} + ... + \beta_{s+1} x_{t-s} + u_t$
			\end{center}
			El efecto acumulado a largo plazo en $y$ cuando $\Delta x$ es:
			\begin{center}
			 	$\beta_1 + \beta_2 + ... + \beta_{s+1}$
			\end{center}
		 	\item \textbf{Modelos dinámicos} - un rezago de la variable dependiente es parte de las variables independientes (endogeneidad). Conceptualmente:
		 	\begin{center}
		 		$y_t = \beta_0 + \beta_1 y_{t-1} + ... + \beta_s y_{t-s} + u_t$
		 	\end{center}
	 		\item Combinaciones de lo anterior, como modelos de rezagos distribuidos racionales (rezagos distribuidos + dinámicos).
		\end{itemize}
% end of column 1 page 1
\columnbreak
% start of column 2 page 1
\section*{Supuestos y propiedades}
	\subsection*{Supuestos MCO bajo series temporales}
		Bajo estos supuestos, los estimadores de los parámetros MCO presentarán buenas propiedades. \textbf{Supuestos Gauss-Markov extendidos en series temporales}:
		\begin{enumerate}[leftmargin=*, label=st\arabic*.]
			\item \textbf{Linealidad de parámetros y dependencia débil}.
			\begin{enumerate}[leftmargin=*, label=\alph*.]
				\item $y_t$ debe ser una función lineal de $\beta$'s.
				\item El proceso estocástico $\lbrace(x_t, y_t): t = 1, 2, ..., T\rbrace$ es estacionario y débilmente dependiente.
			\end{enumerate} 
			\item \textbf{No colinealidad perfecta}.
			\begin{itemize}[leftmargin=*]
				\item No hay variables independientes que sean constantes: $Var(x_j) \neq 0$
				\item No hay una relación lineal exacta entre variables independientes.
			\end{itemize}
			\item \textbf{Media condicional cero y correlación cero}.
			\begin{enumerate}[leftmargin=*, label=\alph*.]
				\item No hay errores sistemáticos: $E(u_t | x_{1t}, ..., x_{kt}) = E(u_t) = 0 \rightarrow$ \textbf{exogeneidad fuerte} (a implica b).
				\item No hay variables relevantes no incluidas en el modelo: $Cov(x_{jt} , u_t) = 0$ para cualquier $j = 1, ..., k \rightarrow$ \textbf{exogeneidad débil}.
			\end{enumerate}
			\item \textbf{Homocedasticidad}. Var. de los residuos es igual para cualquier nivel de $x$: $Var(u_t | x_{1t}, ..., x_{kt}) = \sigma^2$
			\item \textbf{No autocorrelación}. Los residuos no contienen información sobre otros residuos: $Corr(u_t, u_s | x) = 0$ para cualquier $t \neq s$.
			\item \textbf{Normalidad}. Los residuos son independientes e idénticamente distribuidos (\textbf{i.i.d.}): $u \sim N(0,\sigma^2)$
			\item \textbf{Tamaño de datos}. El número de observaciones disponibles debe ser mayor a $(k + 1)$ parámetros a estimar. (YA satisfecho bajo situaciones asintóticas)
		\end{enumerate}	
	\subsection*{Propiedades asintóticas de MCO}
		Bajo los supuestos del modelo econométrico y el Teorema Central del Límite:
		\begin{itemize}[leftmargin=*]
			\item De (1) a (3a): MCO es \textbf{insesgado}. $E(\hat{\beta}_j) = \beta_j$
			\item De (1) a (3): MCO es \textbf{consistente}. $plim(\hat{\beta}_j) = \beta_j$ (a (3b) sin (3a), exogeneidad débil, insesg. y consistente).
			\item De (1) a (5): \textbf{normalidad asintótica} de MCO (entonces, (6) es necesariamente satisfecho): $u \sim_a N(0,\sigma^2)$.
			\item De (1) a (5): \textbf{estimador insesgado de $\sigma^2$}. $E(\hat{\sigma}^2) = \sigma^2$
			\item De (1) a (5): MCO es \textcolor{blue}{\href{https://www.youtube.com/watch?v=68ugkg9RePc}{MELI}} (Mejor Estimador Lineal Insesgado) or \textbf{eficiente}. 
			\item De (1) a (6): contrastes de hipótesis e intervalos de confianza son fiables.
		\end{itemize}
% end of column 2 page 1
\columnbreak
% start of column 3 page 1
\section*{Tendencia y estacionalidad}
	\textbf{Regresión espuria} - es cuando la relación entre $y$ y $x$ es debida a factores que afectan a $y$ y que tienen correlación con $x$, $Corr(x, u) \neq 0$. Es el  \textbf{incumplimiento de st3}.
	\subsection*{Tendencia}
		Dos series temporales pueden tener la misma (o contraria) tendencia, lo que lleva a altos niveles de correlación. Esto provoca una falsa apariencia de causalidad, el problema es \textbf{regresión espuria}. Dado el modelo:
		\begin{center}
			$y_t = \beta_0 + \beta_1 x_t + u_t$
		\end{center}
		donde:
		\begin{center}
			$y_t = \alpha_0 + \alpha_1 Tendencia + v_t$
			\\ $x_t = \gamma_0 + \gamma_1 Tendencia + v_t$
		\end{center}
		Añadir una tendencia al modelo puede resolver el problema:
		\begin{center}
			$y_t = \beta_0 + \beta_1 x_t + \beta_2 Tendencia + u_t$
		\end{center}
		Una tendencia puede ser lineal o no lineal (cuadrática, cúbica, exponencial, etc.)
		\\ Otra manera, es hacer uso del \textbf{filtro Hodrick-Prescott} y extraer la tendencia (suavizado) y el componente cíclico.
	\subsection*{Estacionalidad}
		\setlength{\multicolsep}{0pt} % reduce vertical spacing betwen subsection and multicols
		\begin{multicols}{2} % set columns to 2
		% start of text column
			Una serie temporal puede manifestar estacionalidad. Esto es, que la serie está sujeta a variaciones estacionales o patrones usualmente relacionados al clima.
			\\ Por ejemplo, el PIB (negro) es usualmente mayor en verano y menor en invierno. Serie ajustada estacionalmente ({\color{red} rojo}) en comparación.
		% end of text column
		\columnbreak
		% start of graph column
			\begin{tikzpicture}[scale=0.18]
				\draw [step=2, gray, very thin] (-10,-10) grid (10,10); % background grid
				\draw [thick, <->] (-10,10) node [anchor=south] {$y$} -- (-10,-10) -- (10,-10) node [anchor=north] {$t$}; %axis
				\draw [black]
				(-8.0000,-7.2061) -- 
				(-7.5676,-5.1908) -- 
				(-7.1351,-8.0000) -- 
				(-6.7027,-2.3817) -- 
				(-6.2703,-3.9695) -- 
				(-5.8378,-1.1603) -- 
				(-5.4054,-4.5802) -- 
				(-4.9730,0.8550) -- 
				(-4.5405,-1.5267) -- 
				(-4.1081,-0.3053) -- 
				(-3.6757,-4.5191) -- 
				(-3.2432,-0.4885) -- 
				(-2.8108,-2.3206) -- 
				(-2.3784,-0.4275) -- 
				(-1.9459,-4.2137) -- 
				(-1.5135,0.3664) -- 
				(-1.0811,-1.7099) -- 
				(-0.6486,-0.5496) -- 
				(-0.2162,-4.3969) -- 
				( 0.2162, 1.0992) -- 
				(0.6486,-1.0382) -- 
				(1.0811,1.2824) -- 
				(1.5135,-2.8702) -- 
				(1.9459,1.7099) -- 
				(2.3784,-1.0382) -- 
				(2.8108,1.5267) -- 
				(3.2432,-1.8321) -- 
				(3.6757,3.3588) -- 
				(4.1081,1.0992) -- 
				(4.5405,4.2137) -- 
				(4.9730,0.9160) -- 
				(5.4054,6.2901) -- 
				(5.8378,4.3969) -- 
				(6.2703,6.5954) -- 
				(6.7027,3.4198) -- 
				(7.1351,8.0000) -- 
				(7.5676,6.1069) -- 
				(8.0000,6.9008);
				\draw [red] 
				(-8.0000,-6.2061) -- 
				(-7.5676,-6.0018) -- 
				(-7.1351,-6.1000) -- 
				(-6.7027,-5.0817) -- 
				(-6.2703,-3.9095) -- 
				(-5.8378,-3.0603) -- 
				(-5.4054,-3.0002) -- 
				(-4.9730,-2.4550) -- 
				(-4.5405,-2.5267) -- 
				(-4.1081,-2.3053) -- 
				(-3.6757,-2.5191) -- 
				(-3.2432,-2.4885) -- 
				(-2.8108,-2.3206) -- 
				(-2.3784,-2.4275) -- 
				(-1.9459,-2.2137) -- 
				(-1.5135,-1.6664) -- 
				(-1.0811,-2.0099) -- 
				(-0.6486,-1.8496) -- 
				(-0.2162,-1.3969) -- 
				(0.2162,-1.0992) -- 
				(0.6486,-1.0382) -- 
				(1.0811,-1.2824) -- 
				(1.5135,-1.0002) -- 
				(1.9459,-0.8099) -- 
				(2.3784,-0.6382) -- 
				(2.8108,-0.6267) -- 
				(3.2432,0.6321) -- 
				(3.6757,1.0588) -- 
				(4.1081,1.3992) -- 
				(4.5405,2.2137) -- 
				(4.9730,2.5160) -- 
				(5.4054,3.5901) -- 
				(5.8378,4.0969) -- 
				(6.2703,5.2954) -- 
				(6.7027,5.3198) -- 
				(7.1351,6.2000) -- 
				(7.5676,6.9069) -- 
				(8.0000,6.6008);
			\end{tikzpicture}
		% end of graph column
		\end{multicols}
		\begin{itemize}[leftmargin=*]
			\item Este problema es \textbf{regresión espuria}. Un ajuste estacional puede solucionarlo.
		\end{itemize}
		Un \textbf{ajuste estacional} sencillo es crear variables estacionales binarias y añadirlas al modelo. Por ejemplo, una serie trimestral ($QX_t$ son variables binarias):
		\begin{center}
			$y_t = \beta_0 + \beta_1 Q2_t + \beta_2 Q3_t + \beta_3 Q4_t + \beta_4 x_{1t} + ... + \beta_k x_{kt} + u_t$
		\end{center}
		Otro método es ajustar estacionalmente (sa) las variables, y entonces, hacer la regresión con las variables ajustadas:
		\begin{center}
			$z_t = \beta_0 + \beta_1 Q2_t + \beta_2 Q3_t + \beta_3 Q4_t  + v_t \rightarrow \hat{v}_t + E(z_t) = \hat{z}_t^{sa}$
			$\hat{y}_t^{sa} = \beta_0 + \beta_1 \hat{x}_{1t}^{sa} + ... + \beta_k \hat{x}_{kt}^{sa} + u_t$
		\end{center}
		Hay métodos mucho mejores y complejos para ajustar estacionalmente, como el \textbf{X-13ARIMA-SEATS}.
% end of column 3 page 1	
\columnbreak
% start of column 1 page 2
\section*{Autocorrelación}
	El residuo de cualquier observación, $u_t$, está correlacionado con el residuo de cualquier otra observación. Las observaciones no son independientes. Es el \textbf{incumplimiento de st5}.
	\begin{center}
		$Corr(u_t, u_s | x) \neq 0$ para cualquier $t \neq s$
	\end{center}
	\subsection*{Consecuencias}
		\begin{itemize}[leftmargin=*]
			\item Estimadores MCO son insesgados.
			\item Estimadores MCO son consistentes.
			\item MCO ya \textbf{no es eficiente}, pero sigue siendo ELI (Estimador Lineal Insesgado).
			\item La \textbf{estimación de la varianza de los estimadores es sesgada}: la construcción de intervalos de confianza y contraste de hipótesis no son fiables.
		\end{itemize}
	\subsection*{Detección}
		\begin{itemize}[leftmargin=*]
			\item \textbf{Gráficos de dispersión} - buscar patrones de dispersión en $u_{t-1}$ vs. $u_t$.
			\setlength{\multicolsep}{0pt} % reduce vertical spacing betwen subsection and multicols
			\setlength{\columnsep}{6pt} % increment spacing between columns
			% scatter plots
			\begin{multicols}{3} % set columns to 3
			% start of Ac. plot
				\begin{center}
					\textbf{\footnotesize Ac.}
				\end{center}
				\vspace{2.0pt}
				\begin{tikzpicture}[scale=0.11]
					\draw [step=2, gray, very thin] (-10,-10) grid (10,10); % grid
					\draw [thick,->] (-10,0) -- (10,0) node [anchor=south] {$u_{t-1}$}; % ut-1 axis
					\draw [thick,-] (-10,-10) -- (-10,10) node [anchor=west] {$u_t$}; % ut axis
					\draw plot [only marks, mark=*, mark size=6, domain=-8:8, samples=50] (\x,{rnd * 6 + (-2 * (\x)^2 + 40) * 0.1}); % data points
					\draw [thick, dashed, red, -latex] plot [domain=-8:8] (\x,{3 + (-2 * (\x)^2 + 40) * 0.1}); % red arrow
				\end{tikzpicture}
			% end of Ac. plot
			\columnbreak
			% start of Ac.(+) plot
				\begin{center}
					\textbf{\footnotesize Ac.(+)}
				\end{center}
				\begin{tikzpicture}[scale=0.11]
					\draw [step=2, gray, very thin] (-10,-10) grid (10,10); % grid
					\draw [thick,->] (-10,0) -- (10,0) node [anchor=north] {$ u_{t-1}$}; % ut-1 axis
					\draw [thick,-] (-10,-10) -- (-10,10) node [anchor=west] {$u_t$}; % ut axis
					\draw plot [only marks, mark=*, mark size=6, domain=-8:8, samples=25] (\x,{rnd * 6 + 0.5 * \x - 3}); % data points
					\draw [thick, dashed, red, -latex] plot [domain=-8:8] (\x,{3 + 0.5 * \x - 3}); % red arrow
				\end{tikzpicture}
			% end of Ac.(+) plot	
			\columnbreak
			% start of Ac.(-) plot
				\begin{center}
					\textbf{\footnotesize Ac.(-)}
				\end{center}
				\begin{tikzpicture}[scale=0.11]
					\draw [step=2, gray, very thin] (-10,-10) grid (10,10); % grid
					\draw [thick,->] (-10,0) -- (10,0) node [anchor=south] {$u_{t-1}$}; % ut-1 axis
					\draw [thick,-] (-10,-10) -- (-10,10) node [anchor=west] {$u_t$}; % ut axis
					\draw plot [only marks, mark=*, mark size=6, domain=-8:8, samples=25] (\x,{rnd * 6 - 0.5 * \x - 3}); % data points
					\draw [thick, dashed, red, -latex] plot [domain=-8:8] (\x,{3 - 0.5 * \x - 3}); % red arrow
				\end{tikzpicture}
			% end of Ac.(-) plot
			\end{multicols}
			% correlogram
			\begin{multicols}{2} % set columns to 2
			% start of text column 1
				\item\textbf{Correlograma} - compuesto de la función de autocorrelación (FAC) y el FAC parcial (FACP).
			% end of text column 1
			\columnbreak
			% start of text column 2
				\begin{itemize}[leftmargin=*]
					\item Eje Y: correlación [-1,1].
					\item Eje X: número de retardo.
					\item Líneas azules: $\pm 1.96/T^{0.5}$
				\end{itemize}
			% end of text column 2
			\end{multicols}
			\begin{center}
				\begin{tikzpicture}[scale=0.22]
					% acf plot
					\node at (17.5,23) {\tiny \textbf{FAC}};
					\node at (-1.5,22) {\tiny 1};
					\node at (-1.5,17) {\tiny 0};
					\node at (-1.5,12) {\tiny -1};
					\draw [step=2, gray, very thin] (0,12) rectangle (35,22); % acf
					\draw [thick,|->] (0,22) -- (0,12) -- (35,12); % axis
					\fill [red] (1.5,17) rectangle (2,21.9);
					\fill [red] (3.5,17) rectangle (4,21.8);
					\fill [red] (5.5,17) rectangle (6,21.7);
					\fill [red] (7.5,17) rectangle (8,21.6);
					\fill [red] (9.5,17) rectangle (10,21.5);
					\fill [red] (11.5,17) rectangle (12,21.5);
					\fill [red] (13.5,17) rectangle (14,21.4);
					\fill [red] (15.5,17) rectangle (16,21.3);
					\fill [red] (17.5,17) rectangle (18,21.2);
					\fill [red] (19.5,17) rectangle (20,21);
					\fill [red] (21.5,17) rectangle (22,20.9);
					\fill [red] (23.5,17) rectangle (24,20.7);
					\fill [red] (25.5,17) rectangle (26,20.5);
					\fill [red] (27.5,17) rectangle (28,20.3);
					\fill [red] (29.5,17) rectangle (30,20.1);
					\fill [red] (31.5,17) rectangle (32,19.8);
					\fill [red] (33.5,17) rectangle (34,19.6);
					\draw [blue, very thin] (0,18.5) -- (35,18.5); % above blue line
					\draw [dashed, very thin] (0,17) -- (35,17); % center blue dashed line
					\draw [blue, very thin] (0,15.5) -- (35,15.5); % below blue line
					% pacf plot
					\node at (17.5,11) {\tiny \textbf{FACP}};
					\node at (-1.5,10) {\tiny 1};
					\node at (-1.5,5) {\tiny 0};
					\node at (-1.5,0) {\tiny -1};
					\draw [step=2, gray, very thin] (0,0) rectangle (35,10); % pacf
					\draw [thick,|->] (0,10) -- (0,0) -- (35,0); % axis
					\fill [red] (1.5,5) rectangle (2,9.9);
					\fill [red] (3.5,5) rectangle (4,4.8);
					\fill [red] (5.5,5) rectangle (6,4.6);
					\fill [red] (7.5,5) rectangle (8,4.5);
					\fill [red] (9.5,5) rectangle (10,4.5);
					\fill [red] (11.5,5) rectangle (12,4.7);
					\fill [red] (13.5,5) rectangle (14,4.5);
					\fill [red] (15.5,5) rectangle (16,4.8);
					\fill [red] (17.5,5) rectangle (18,4.6);
					\fill [red] (19.5,5) rectangle (20,4.9);
					\fill [red] (21.5,5) rectangle (22,4.7);
					\fill [red] (23.5,5) rectangle (24,4.6);
					\fill [red] (25.5,5) rectangle (26,4.8);
					\fill [red] (27.5,5) rectangle (28,4.9);
					\fill [red] (29.5,5) rectangle (30,4.8);
					\fill [red] (31.5,5) rectangle (32,4.8);
					\fill [red] (33.5,5) rectangle (34,4.9);
					\draw [blue, very thin] (0,6.5) -- (35,6.5); % above blue line
					\draw [dashed, very thin] (0,5) -- (35,5); % center blue dashed line
					\draw [blue, very thin] (0,3.5) -- (35,3.5); % below blue line
				\end{tikzpicture}
			\end{center}
			Conclusiones difieren entre procesos de autocorrelación.
% end of column 1 page 2
\columnbreak
% start of column 2 and 3 page 2
			\begin{itemize}[leftmargin=*]
				\item \textbf{Proceso MA($q$)}. \textbf{FAC}: sólo los primeros $q$ coeficientes son significativos. El resto se anulan bruscamente. \textbf{FACP}: decrecimiento rápido exponencial atenuado u ondas sinusoidales.
				\item \textbf{Proceso AR($p$)}. \textbf{FAC}: decrecimiento rápido exponencial atenuado u ondas sinusoidales. \textbf{FACP}: sólo los primeros $p$ coeficientes son significativos. El resto se anulan bruscamente.
				\item \textbf{Proceso ARMA($p$,$q$)}. \textbf{FAC}: los coeficientes no se anulan bruscamente y presentan un decrecimiento rápido. \textbf{FACP}: los coeficientes no se anulan bruscamente y presentan un decrecimiento rápido.
			\end{itemize}
			Si los coeficientes de la FAC no decaen rápidamente, hay claro indicio de falta de estacionariedad en media, lo que llevaría a tomar primeras diferencias en la serie original.
			\item \textbf{Formal tests} - $H_0$: No auto-correlation. 
			\\ Suponiendo que $u_t$ sigue un proceso AR(1):
			\begin{center}
				$u_t = \rho_1 u_{t-1} + \varepsilon_t$
			\end{center}
			donde $\varepsilon_t$ es ruido blanco.
			\begin{itemize}[leftmargin=*]
				\item \textbf{Prueba t AR(1)} (regresores exógenos):
				\begin{center}
					$t = \frac{\hat{\rho}_1}{ee(\hat{\rho}_1)} \sim t_{T-k-1, \alpha/2}$
				\end{center}
				\begin{itemize}[leftmargin=*]
					\item $H_1$: Autocorrelación de orden uno, AR(1).
				\end{itemize}
				\item \textbf{Estadístico Durbin-Watson} (regresores exógenos y normalidad de residuos):
				\begin{center}
					$d = \frac{\sum_{t=2}^n (\hat{u}_t - \hat{u}_{t-1})^2}{\sum_{t=1}^n \hat{u}_t^2} \approx 2 (1 - \hat{\rho}_1)$  ,  $0 \leq d \leq 4$
				\end{center}
				\begin{itemize}[leftmargin=*]
					\item $H_1$: Autocorrelación de orden uno, AR(1).
				\end{itemize}
				\begin{center}
					\scalebox{0.8}{
						\begin{tabular}{| c | c | c | c |}
							\hline
							$d = $ & 0 & 2 & 4 \\
							\hline
							$\rho \approx$ & -1 & 0 & 1 \\
							\hline
						\end{tabular}
					}
					\begin{tikzpicture}[scale=0.28] % durbin-watson test plot
						\node at (-1.5,9.5) {\scalebox{1.2}{\tiny $f(d)$}};
						\draw [thick] (0,10) -- (0,0) -- (25,0); % axis
						\draw [very thin, dashed] (12.5,0) -- (12.5,10);
						\fill [pattern=north west lines, pattern color=black] (5,0)  rectangle  (9,10);
						\draw (5,0) -- (5,10);
						\draw (9,0) -- (9,10);
						\fill [pattern=north west lines, pattern color=black] (16,0) rectangle (20,10);
						\draw (16,0) -- (16,10);
						\draw (20,0) -- (20,10);
						% x axis labels
						\node at (0,-0.6) {\scalebox{1.2}{\tiny 0}};
						\node at (5,-0.6) {\scalebox{1.2}{\tiny $d_L$}};
						\node at (9,-0.6) {\scalebox{1.2}{\tiny $d_U$}};
						\node at (12.5,-0.6) {\scalebox{1.2}{\tiny 2}};
						\node at (16.7,-1) {\scalebox{1.1}{\tiny \rotatebox{-20}{$4 - d_U$}}};
						\node at (20.7,-1) {\scalebox{1.1}{\tiny \rotatebox{-20}{$4 - d_L$}}};
						\node at (25,-0.6) {\scalebox{1.2}{\tiny 4}};
						% text inside plot
						\node at (2.5,5.5) {\scalebox{1.2}{\tiny Rech. $H_0$}};
						\node at (2.5,4.5) {\scalebox{1.2}{\tiny AR(+)}};
						\node [text=red] at (7,5) {\scalebox{1.2}{\tiny \rotatebox{-70}{\textbf{INCONCLUYENTE}}}};
						\node at (12.5,5.5) {\scalebox{1.2}{\tiny Aceptar $H_0$}};
						\node at (12.5,4.5) {\scalebox{1.2}{\tiny No AR}};
						\node [text=red] at (18,5) {\scalebox{1.2}{\tiny \rotatebox{-70}{\textbf{INCONCLUYENTE}}}};
						\node at (22.5,5.5) {\scalebox{1.2}{\tiny Rech. $H_0$}};
						\node at (22.5,4.5) {\scalebox{1.2}{\tiny AR(-)}};
					\end{tikzpicture}
				\end{center}
				\item \textbf{h de Durbin} (regresores endógenos):
				\begin{center}
					$h = \hat{\rho} \sqrt{\frac{T}{1 - T \times \upsilon}}$
				\end{center}
				donde $\upsilon$ es la varianza estimada del coeficiente asociado a la variable endógena.
				\begin{itemize}[leftmargin=*]
					\item $H_1$: Autocorrelación de orden uno, AR(1).
				\end{itemize}
				\item \textbf{Prueba Breusch-Godfrey} (regresores endógenos): puede detectar procesos MA($q$) y AR($p$) ($\varepsilon_t$ ruido b.):
				\begin{itemize}[leftmargin=*]
					\item MA($q$): $u_t = \varepsilon_t - \theta_1 u_{t-1} - ... - \theta_q u_{t-q}$
					\item AR($p$): $u_t = \rho_1 u_{t-1} + ... + \rho_p u_{t-p} + \varepsilon_t$
				\end{itemize}
				Bajo $H_0$: No autocorrelación:
				\begin{center}
					$\hfill T \times R^2_{\hat{u}_t} \sim_a \chi^2_q \hfill \textbf{or} \hfill T \times R^2_{\hat{u}_t} \sim_a \chi^2_p \hfill$
				\end{center}
				\begin{itemize}[leftmargin=*]
					\item $H_1$: Autocorrelación de orden $q$ (ó $p$).
				\end{itemize}
				\item \textbf{Prueba Ljung-Box Q}:
				\begin{itemize}[leftmargin=*]
					\item $H_1$: Existe autocorrelación.
				\end{itemize}
			\end{itemize}
		\end{itemize}
	\subsection*{Corrección}
		\begin{itemize}[leftmargin=*]
			\item Usar MCO con un estimador de la \textbf{matriz de varianzas-covarianzas} \textbf{robusto a la autocorrelación}, por ejemplo, la propuesta de \textbf{Newey-West}.
			\item Usar \textbf{Mínimos Cuadrados Generalizados}. Suponiendo $y_t = \beta_0 + \beta_1 x_t + u_t$, con $u_t = \rho u_{t-1} + \varepsilon_t$, donde $|\rho| < 1$ y $\varepsilon_t$ es ruido blanco.
			\begin{itemize}[leftmargin=*]
				\item Si \textbf{$\rho$ es conocido, crear un modelo cuasi-diferenciado}:
				\begin{center}
					$y_t - \rho y_{t-1} = \beta_0 (1 - \rho) + \beta_1 (x_t - \rho x_{t-1}) + u_t - \rho u_{t-1}$
					\ $y_t^* = \beta_0^* + \beta_1 x_t^* + u_t^*$
				\end{center}
				donde $u_t^*$ es ruido blanco, y estimarlo por MCO.
				\item Si \textbf{$\rho$ es desconocido, estimarlo} -por ejemplo- \textbf{el método de Cochrane-Orcutt} (el método de Prais-Winsten también es bueno):
				\begin{enumerate}[leftmargin=*]
					\item Obtener $\hat{u}_t$ del modelo original.
					\item Estimar $\hat{u}_t = \rho \hat{u}_{t-1} + \varepsilon_t$ y obtener $\hat{\rho}$.
					\item Crear un modelo cuasi-diferenciado:
					\begin{center}
						$y_t - \hat{\rho} y_{t-1} = \beta_0 (1 - \hat{\rho}) + \beta_1 (x_t - \hat{\rho} x_{t-1}) + u_t - \hat{\rho} u_{t-1}$
						\ $y_t^* = \beta_0^* + \beta_1 x_t^* + u_t^*$
					\end{center}
					donde $u_t^*$ es ruido blanco, y estimarlo por MCO.
					\item Obtener $\hat{u}_t^*$ y repetir desde el paso 2.
					\item El método termina cuando los parámetros estimados varían muy poco entre iteraciones.
				\end{enumerate}
			\end{itemize}
		\item Si \textbf{no se arregla, buscar fuerte depend.} en la serie.
		\end{itemize}

\section*{Estacionariedad y dependencia débil}
	Estacionariedad es estabilidad de las distribuciones conjuntas de probabilidad de un proceso a medida que este progresa el tiempo. Permite identificar correctamente las relaciones -inalteradas en el tiempo- entre variables.
	\subsection*{Procesos estacionarios y no estacionarios}
		\begin{itemize}[leftmargin=*]
			\item \textbf{Proceso estacionario} (estacionariedad fuerte) - la dist. de prob. es estable en el tiempo: si se toma cualquier colección de variables aleatorias, y se mueven $h$ periodos, la dist. conjunta de prob. debe permanecer inalterada.
% end of column 2 and 3 page 2
\columnbreak
% start of column 1 page 3
			\item \textbf{Proceso no estacionario} - por ejemplo, una serie con tendencia, donde al menos la media cambia con el tiempo.
			\item \textbf{Proceso estacionario en covarianza} - es una forma más débil de estacionariedad:
			\begin{itemize}[leftmargin=*]
				\item $E(x_t)$ es constante.
				\item $Var(x_t)$ es constante.
				\item Para cualquier $t$,  $h \geq 1$, la $Cov(x_t, x_{t+h})$ depende sólo de $h$, no de $t$.
			\end{itemize}
		\end{itemize}
	\subsection*{Series temporales de dependencia débil}
		Es importante porque reemplaza el requisito de muestreo aleatorio, dando por supuesto la validez del Teorema Central del Límite (requiere estacionariedad y una forma de dependencia débil). Los procesos débilmente dependientes \textbf{también se conocen como, I(0)}.
		\begin{itemize}[leftmargin=*]
			\item \textbf{Dependencia débil} - restringe cuán cercana la relación entre $x_t$ y $x_{t+h}$ puede ser a medida que la distancia temporal entre las series aumenta ($h$).
		\end{itemize}
		Un \textbf{proceso estacionario} $\lbrace x_t: t = 1, 2, ..., T \rbrace$ es débilmente dependiente cuando $x_t$ y $x_{t+h}$ son casi independientes a medida que $h$ aumenta sin límite.
		\\ Un \textbf{proceso estacionario en covarianza} es débilmente dependiente si la correlación entre $x_t$ y $x_{t+h}$ tiende a $0$ lo suficientemente rápido cuando $h \rightarrow \infty$ (no están asintóticamente correlacionados).
		\\ Algunos ejemplos de series estacionarias y débilmente dependientes:
		\begin{itemize}[leftmargin=*]
			\item \textbf{Media móvil} - $\lbrace x_t \rbrace$ es una media móvil de orden uno MA($q=1$):
			\begin{center}
				$x_t = e_t + \theta_1 e_{t-1}$
			\end{center}
			donde $\lbrace e_t: t = 0, 1, ..., T \rbrace$ es una secuencia \textsl{i.i.d.} con media cero y varianza $\sigma^2_e$.
			\item \textbf{Proceso autorregresivo} - $\lbrace x_t \rbrace$ es un proceso autorregresivo de orden uno AR($p=1$):
			\begin{center}
				$x_t = \rho_1 x_{t-1} + e_t$
			\end{center}
			donde $\lbrace e_t: t = 0, 1, ..., T \rbrace$ es una secuencia \textsl{i.i.d.} con media cero y varianza $\sigma^2_e$.
			\\ Si $|\rho_1|<1$, entonces $\lbrace x_t \rbrace$ es un proceso AR(1) que es débilmente dependiente. Es estacionario en covarianza, $Corr(x_t, x_{t-1}) = \rho_1$.
			\item \textbf{Proceso ARMA} - es una combinación de los dos anteriores. $\lbrace x_t \rbrace$ es un ARMA($p=1$,$q=1$):
			\begin{center}
				$x_t = e_t + \theta_1 e_{t-1} + \rho_1 x_{t-1}$
			\end{center}
		\end{itemize}
		Una serie con tendencia no puede ser estacionaria, pero puede ser débilmente dependiente (y estacionaria si la serie es filtrada de tendencia).
% end of column 1 page 3
\columnbreak
% start of column 2 page 3
\section*{Series temporales de depend. fuerte}
	La mayoría del tiempo, las series económicas presentan dependencia fuerte (o fuerte persistencia temporal). Algunos casos especiales de procesos de raíz unitaria, I(1):
	\begin{itemize}[leftmargin=*]
		\item \textbf{Paseo aleatorio} - un proceso AR(1) con $\rho_1 = 1$.
		\begin{center}
			$y_t = y_{t-1} + e_t$
		\end{center}
		donde $\lbrace e_t : t = 1, 2, ..., T \rbrace$ es una secuencia \textsl{i.i.d.} con media cero y varianza $\sigma^2_e$ (la última cambia con el tiempo).
		\\ El proceso no es estacionario, es persistente.
		\item \textbf{Paseo aleatorio con \href{https://www.youtube.com/watch?v=pS5d77DQHOI}{deriva}} - un proceso AR(1) con $\rho_1 = 1$ y una constante.
		\begin{center}
			$y_t = \alpha + y_{t-1} + e_t$
		\end{center}
		donde $\lbrace e_t : t = 1, 2, ..., T \rbrace$ es una secuencia \textsl{i.i.d.} con media cero y varianza $\sigma^2_e$.
		\\ El proceso no es estacionario, es persistente.
	\end{itemize}
	\subsection*{Detección de I(1)}
		\begin{itemize}[leftmargin=*]
			\item \textbf{Prueba aumentada de Dickey-Fuller (ADF)} - donde $H_0$: el proceso es raíz unitaria, I(1).
			\item \textbf{Prueba Kwiatkowski–Phillips–Schmidt–Shin (KPSS)} - donde $H_0$: el proceso no es raíz unitaria, I(0).
		\end{itemize}
	\subsection*{Transformar raíz unitaria a depend. débil}
		Los procesos de raíz unitaria son \textbf{integrados de orden uno}, I(1). Esto significa que \textbf{la primera diferencia del proceso es débilmente dependiente} ó I(0) (y usualmente, estacionaria). Por ejemplo, un paseo aleatorio:
		\begin{multicols}{2} % set columns to 2
		% start of text column
			\begin{center}
				$\Delta y_t = y_t - y_{t-1} = e_t$
			\end{center}
			donde $\lbrace e_t \rbrace = \lbrace \Delta y_t \rbrace$  es \textsl{i.i.d.}
			\\ 
			\\ Tomar la primera diferencia de una serie también elimina su tendencia.
			\\ Por ejemplo, una serie con tendencia (negro), y su primera diferencia ({\color{red} rojo}).
		% end of text column
		\columnbreak
		% start of graph column
			\begin{tikzpicture}[scale=0.18]
				\draw [step=2, gray, very thin] (-10,-10) grid (10,10); % background grid
				\draw [thick, <->] (-10,10) node [anchor=south] {$y$} -- (-10,-10) -- (10,-10) node [anchor=north] {$t$}; %axis
				\draw [black]
				(-8,-8) -- 
				(-7.6,-7.5411) -- 
				(-7.2,-7.2841) -- 
				(-6.8,-6.7952) -- 
				(-6.4,-6.4292) -- 
				(-6,-6.0486) -- 
				(-5.6,-5.9534) -- 
				(-5.2,-5.4861) -- 
				(-4.8,-5.2817) -- 
				(-4.4,-4.8405) -- 
				(-4,-4.3267) -- 
				(-3.6,-4.0131) -- 
				(-3.2,-3.7582) -- 
				(-2.8,-3.5296) -- 
				(-2.4,-3.0564) -- 
				(-2,-2.8965) -- 
				(-1.6,-2.4169) -- 
				(-1.2,-1.9134) -- 
				(-0.8,-1.8881) -- 
				(-0.4,-1.1661) -- 
				(0,-0.53) -- 
				(0.4,-0.2821) -- 
				(0.8,0.0327) -- 
				(1.2,0.4915) -- 
				(1.6,0.7483) -- 
				(2,0.8053) -- 
				(2.4,1.0167) -- 
				(2.8,1.4391) -- 
				(3.2,1.8107) -- 
				(3.6,2.2473) -- 
				(4,2.6688) -- 
				(4.4,3.052) -- 
				(4.8,3.5867) -- 
				(5.2,4.322) -- 
				(5.6,4.9132) -- 
				(6,5.7041) -- 
				(6.4,6.0819) -- 
				(6.8,6.4316) -- 
				(7.2,6.8585) -- 
				(7.6,7.5108) -- 
				(8,8);
				\draw [red]
				(-7.6,1.2835) -- 
				(-7.2,-2.7995) -- 
				(-6.8,1.8898) -- 
				(-6.4,-0.5953) -- 
				(-6,-0.2992) -- 
				(-5.6,-6.0745) -- 
				(-5.2,1.4544) -- 
				(-4.8,-3.8646) -- 
				(-4.4,0.9266) -- 
				(-4,2.3932) -- 
				(-3.6,-1.6553) -- 
				(-3.2,-2.8432) -- 
				(-2.8,-3.3737) -- 
				(-2.4,1.5724) -- 
				(-2,-4.7658) -- 
				(-1.6,1.7033) -- 
				(-1.2,2.1863) -- 
				(-0.8,-7.4877) -- 
				(-0.4,6.6075) -- 
				(0,4.869) -- 
				(0.4,-2.9853) -- 
				(0.8,-1.6322) -- 
				(1.2,1.2832) -- 
				(1.6,-2.8046) -- 
				(2,-6.8477) -- 
				(2.4,-3.7232) -- 
				(2.8,0.547) -- 
				(3.2,-0.4838) -- 
				(3.6,0.8346) -- 
				(4,0.5268) -- 
				(4.4,-0.2468) -- 
				(4.8,2.816) -- 
				(5.2,6.8759) -- 
				(5.6,3.9619) -- 
				(6,8) -- 
				(6.4,-0.3568) -- 
				(6.8,-0.9251) -- 
				(7.2,0.6366) -- 
				(7.6,5.1971) -- 
				(8,1.8972);
			\end{tikzpicture}
		% end of graph column
		\end{multicols}
		Cuando una serie I(1) es estrictamente positiva, se suele transformar a logaritmos antes de tomar primeras diferencias. Esto es, para obtener el cambio porcentual (aprox.) de la serie:
		\begin{center}
			$\Delta log(y_t) = log(y_t) - log(y_{t-1}) \approx (y_t - y_{t-1}) / y_{t-1}$
		\end{center}
% end of column 2 page 3
\columnbreak
% start of column 3 page 3
\section*{Cointegración}
	Cuando \textbf{dos series son I(1), pero una combinación lineal de estas es I(0)}. Si es el caso, la regresión de una serie sobre la otra no es espuria, sino que expresa algo sobre la relación a largo plazo.
	\\ Por ejemplo: $\lbrace x_t \rbrace$ y $\lbrace y_t \rbrace$ son I(1), pero $y_t - \beta x_t = u_t$ donde $\lbrace u_t \rbrace$ es I(0). ($\beta$ toma el nombre de parámetro cointegrador).

\section*{Heterocedasticidad en series temp.}
	\textbf{Afecta al supuesto st4}, lo que lleva a que \textbf{MCO no sea eficiente}. 
	\\ Algunas pruebas que funcionan pueden ser la de Breusch-Pagan o la de White, donde $H_0$: No heterocedasticidad. Es \textbf{importante que no haya autocorrelación para el correcto funcionamiento de las pruebas} (así que, primero es necesario probar la existencia de autocorrelación).
	\subsection*{ARCH}
		La heterocedasticidad condicional autorregresiva (ARCH), es un modelo para analizar una forma de heterocedasticidad dinámica, donde la varianza del error sigue un proceso AR($p$).
		\\ Dado el modelo:
		\begin{center}
			$y_t = \beta_0 + \beta_1 z_t + u_t$
		\end{center}
		donde, hay AR(1) y heterocedasticidad:
		\begin{center}
			$E(u^2_t | u_{t-1}) = \alpha_0 + \alpha_1 u^2_{t-1}$
		\end{center}
	\subsection*{GARCH}
		La heterocedasticidad condicional autorregresiva general (GARCH), es un modelo similar a ARCH, pero en este caso, la varianza del error sigue un proceso ARMA($p$,$q$).

\section*{Predicciones}
	Dos tipos de predicciones:
	\begin{itemize}[leftmargin=*]
		\item Del valor medio de $y$ para un valor específico de $x$.
		\item De un valor individual de $y$ para un valor específico de $x$.
	\end{itemize}
	Si los valores de las variables ($x$) se aproximan al valor medio ($\overline{x}$), la amplitud del intervalo de confianza de la predicción será menor.
% end of column 3 page 3
% end of content
\end{multicols}

\end{document}